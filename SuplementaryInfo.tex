\section{Appendix A}


\subsection{Effective SOC Hamiltonian}	
%Probably too much. 
The effective spin-orbit coupling Hamiltonian can be derived from the electric dipole Hamiltonian describing the interaction between our Rb atoms and the multiple frequency Raman lasers


\begin{align}
	\hat{H}_{AL}(t) = \left[\Omega_{21}\cos(2\kr x-\omega_{21} t +\phi_1)+\Omega_{31}\cos(2\kr x-\omega_{31} t+\phi_2)+\Omega_{41}\cos(2\kr x-\omega_{41} t + \phi_3)\right]\hat{F_x}
\end{align}
%
where $\Omega_{ij}\propto \vec{E_i}\times\vec{E_j^{\star}}$ represents the coupling strength associated to each pair of Raman beams and $\omega_{ij} = \omega_{i}-\omega_{j} $. We choose the frequencies so that $\omega_{31} + \omega{21}$ is at 4 photon resonance with the $m_f = +1\rightarrow m_f = -1$ transition. We then apply a rotation about the z axis $\hat{U} = e^{i\bar{\omega} t\hat{F}_z}$ where $\bar{\omega} = \frac{\omega_{21}+\omega_{31}}{2}$. For the choice of parameters 
$\Omega_{21} =\Omega_{31} = \Omega $, $\Omega_{41}=\Omega_0$, and $\omega_{41} =\frac{\omega_{21}+\omega_{31}}{2}$,and after applying the rotating wave approximation (RWA), the Hamiltonian transforms to  
%
%

	
\begin{align}
	\begin{split}
		\hat{\tilde{H}}_{AL}(t) &= \frac{1}{2}\lbrace\Omega\cos(2\kr x+\delta\omega t +\phi_1)+\Omega\cos(2\kr x-\delta\omega t + \phi_2) +\Omega_0\cos(2\kr x + \phi_3)\rbrace \hat{F}_x \\
		& -  \frac{1}{2}\lbrace\Omega\sin(2\kr x+\delta\omega t +\phi_1)+\Omega\sin(2\kr x-\delta\omega t +\phi_2)+\Omega_0\sin(2\kr x +\phi_3)\rbrace \hat{F}_y
	\end{split}
\end{align}

we have the freedom of defining the time origin so we can get rid of one phase, lets make it $\phi_3$ for convenience. Can I get rid of a second phase? Make $\phi_1 = -\phi_2$
 
 %
 %
\begin{align}
 	\begin{split}
 		\hat{\tilde{H}}_{AL}(t)  = [\frac{\Omega_0}{2} + \Omega\cos(\delta\omega t+\phi_1)][\cos(2\kr x)\hat{F}_x - \sin(2\kr x)\hat{F}_y],
 	\end{split}
\end{align}


where we have defined $\delta\omega=\frac{\omega_{31}-\omega_{21}}{2}$. This Hamiltonian term describes a helically precessing effective Zeeman field with amplitude oscillating periodically in time. 

The complete Hamiltonian can therefore be written as

\begin{align}
	\begin{split}
		\hat{H}(t)=\frac{\hbar^2}{2m}\hat{k}^2 + \mu\mathbf{B}_{eff}\cdot\hat{\mathbf{F}}
		\label{Eq:Beff}
	\end{split}
\end{align}

(check units here)
with $\mu\mathbf{B}_{eff}=(\frac{\Omega_0}{2} + \Omega\cos(\delta\omega t+\phi_1))(\cos(2\kr x)\mathbf{e}_x-\sin(2\kr x)\mathbf{e}_y)+\Delta_0\mathbf{e}_z$


We can apply a position dependent rotation $\hat{U} = e^{i2\kr x \hat{F}_z}$ transforms our Hamiltonian into the form of Eq. \ref{Eq:SOCone} with a time dependent Raman coupling.

\begin{align}
	\begin{split}
		\hat{H}(t) = & \frac{\hbar^2}{2m}(\hat{k}-2\kr\hat{F}_z)^2 + (\frac{\Omega_0}{2} + \Omega\cos(\delta\omega t))\hat{F}_x \\& + \epsilon(\hat{F}_z^2-\mathbb{I})+\Delta\hat{F}_z \\
		=&\frac{\hbar^2\hat{k}^2}{2m} + \alpha_0\hat{k}\hat{F}_z +4E_L\mathbb{I} + \frac{\Omega(t)}{2}\hat{F}_x\\
		& +(\epsilon+4E_L)(\hat{F}_z^2-\mathbb{I}) +\Delta_0\hat{F}_z 
	\end{split}
\end{align}

(check factors of 2!)

To get rid of the time dependence in the Hamiltonian and ultimately getting the `tunable' spin-orbit coupling we can choose a transoformation of the Hamiltonian such that $\hat{U}^{\dagger} \frac{\partial\hat{U}}{\partial t} = -i \frac{\Omega(t)}{2}\hat{F}_x$. This will be satisfied for

\begin{align}
	\hat{U} = e^{-i\frac{\Omega}{2}\int_0^t\cos(\delta\omega t')dt'} = e^{-i\frac{\Omega}{2\delta\omega}\sin(\delta\omega t)}.
\end{align}

Under this time dependent transformation, the time evolution of the system will be given by the Hamiltonian
%

\begin{align}
	\begin{split}
		\hat{\tilde{H}} = & \hat{U}^{\dagger}\hat{H}(t)\hat{U} + i\hat{U}^{\dagger} \frac{\partial\hat{U}}{\partial t} \\
		 = &\frac{\hbar^2\hat{k}^2}{2m} + \alpha\hat{k}\hat{F}_z +4E_L\mathbb{I} + \frac{\Omega_0}{2}\hat{F}_x \\
		&+ \frac{\tilde{\Omega}}{2}\hat{F}_{xz} +(\tilde{\epsilon}+4E_L)(\hat{F}_z^2-\mathbb{I}) +\Delta\hat{F}_z 	
	\end{split}
\end{align}
%
%
which is exactly Eq. \ref{Eq:SOCeff}. To arrive to this final form we have transformed the operators that don't commute with $\hat{F}_x$ as

%
\begin{align}
	\begin{split}
		e^{i\theta \hat{F}_x} \hat{F}_z e^{-i\theta \hat{F}_x}=& \cos\theta \hat{F}_z + \sin\theta\hat{F}_y \\
		e^{i\theta \hat{F}_x} \hat{F}_y e^{-i\theta \hat{F}_x} =& -\sin\theta\hat{F}_z +\cos\theta\hat{F}_y \\
		e^{i\theta \hat{F}_x} \hat{F}_z^2 e^{-i\theta \hat{F}_x} = &\cos^2\theta\hat{F}_z^2+\sin^2\theta\hat{F}_y^2 + \sin\theta\cos\theta(\hat{F}_z\hat{F}_y + \hat{F}_y\hat{F}_z).
	\end{split}
\end{align}
%
and neglected the terms oscillating at high frequency

\begin{align}
	\begin{split}
	\cos(\Omega/2\delta\omega\sin(\delta\omega t))&= J_0(\Omega/2\delta\omega) + 2\sum_{n=1}^{\infty}J_{2n}(\Omega/2\delta\omega)\cos(2n(\delta\omega t)) \\
	& \approx J_0(\Omega/2\delta\omega) \\
	\sin(\Omega/2\delta\omega\sin(\delta\omega t))&= 2\sum_{n=0}^{\infty}J_{2n+1}(\Omega/2\delta\omega)\sin((2n+1)(\delta\omega t)) \approx 0,
	\end{split}
\end{align} 


(verify how large $\delta\omega$ needs to be for this approximation to be valid)
%
%
 





