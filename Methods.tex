\section{Methods}

\subsection{Modulated/tripple frequency coupling (theory)}


\subsection{Fourier Spectroscopy}		

The Fourier Spectroscopy technique takes advantage of the time evolution of a bare atomic state after a dressing field is suddenly turned. The initial state becomes a superposition of dressed states and it undergoes Rabi oscillations in time. The spectral components of these oscillations contain information about the energies of the dressed states. 

Aquí algo sobre el caso particular de las energias SOC.

The measurement can be simplified by noticing that a non-moving atom cloud in the laboratory reference frame dressed by a field with non-zero detuning is equivalent to a moving cloud with a resonant field in a suitable moving reference frame. As can also be seen in the Hamiltonian (citarlo aqui)  the detuning term $\delta/Er$  and the momentum term $4 k/k_R$ have the same effect in the energy differences.


For the case of our spin-orbit coupled BECs, the bare state
The system is let to evolve for a finite time T and afterwards the field is snapped off. A Stern-Gerlach pulse applied at our 21 ms time of flight (TOF) allows us to project the state of the condensate back into the bare $m_f$ basis. 





For the experimental sequence we start 


