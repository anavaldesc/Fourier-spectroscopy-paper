\section{Experiment}

We measured the SOC dispersion for various values of $\Omega$ and $\Omega_0$: (i) A time independent spin-orbit coupled system, $\Omega_0\neq0$ and $\Omega=0$, (ii) for a periodically driven spin-orbit coupled system with no DC offset $\Omega\neq0$ and $\Omega_0=0$, and (iii) A time periodically driven spin-orbit coupled system with a DC offset $\Omega\neq0$ and $\Omega_0\neq0$. 

\subsection{Spectroscopy experimental sequence}


We start our experiments with a Rb$^{87}$ Bose-Einstein condensate \cite{lin_rapid_2009} (BEC) with $N\approx 4\times 10^4$ atoms in the $\ket{F=1,m_F=0}$ state, confined in a $1064\nm$ crossed optical dipole trap, with trapping frequencies $(\omega_x,\omega_y,\omega_z)=2\pi(42(3),34(2),133(3))\ \Hz$. We break the degeneracy between the $m_F$ magnetic sub-levels by applying a $17.0556$ G bias field along the z axis, which produces a Zeeman splitting of $12 \ \MHz$. The quadratic Zeeman shift that lowers the energy of the $\ket{F=1,m_F=0}$ state  by $20.9851$ kHz. We adiabatically prepare our BEC in the $\ket{m_F=0}$ by slowly ramping the bias field while applying a $12\MHz$ radio-frequency field. We generate spin-orbit coupling between the magnetic sub levels with a pair of intersecting, cross-polarized Raman beams, with wavelength $\lambda_R=790.024 nm$ propagating along $\mathbf{e}_x+\mathbf{y}$ and $\mathbf{e}_x-\mathbf{e}_y$ as shown in Fig. \ref{fig:Figure1}a. The frequency of the beams is controlled by two acusto-optic modulators (AOMs), one of which is driven at multiple frequencies. When on resonance, the laser frequencies satisfy the condition $\omega_A-\omega_B=\omega_A-\frac{\omega_{B+}+\omega_{B-}}{2}=\omega_Z$, and we change the Raman detuning $\Delta_0$ by keeping the magnetic field constant and changing the value of the frequency $\omega_A$.

We measure Rabi oscillations at fixed $\Delta_O$ by pulsing the Raman beams for time intervals of up to $900\us$. The pulsing times and number of points are chosen so that the bandwidth is comparable or larger than the highest frequency system while maximizing resolution, with the constriction that we start observing decoherence in the oscillations after $900\us$. For the time independent SOC case (i) we chose to pulse for 120 different time intervals, and for the periodically driven SOC cases (ii, iii) we pulse for 180 time intervals. After pulsing the Raman we release our atoms from the optical dipole trap and let them fall for a $21 ms$ time of flight (TOF) time before and apply a spin-dependent force using magnetic field gradient. Our absorption images reveal the atoms spin and momentum distribution, from which we can extract the probability amplitudes by counting the fractional number of atoms in each spin and quasimomentum state. We repeat this procedure for values of Raman detuning within the interval $\pm 12 E_L$ which corresponds to quasimomentum values $\pm 3k_L$.

The time dependent SOC measurements additionally required phase stability between the 3 frequency components in the Raman B field. All the relative phases are set to zero as it maximizes the effective couplings $\Omega$ and $\Omega_0$ for a given Raman field intensity. For a more detailed discussion of the effect of the relative phases in the Hamiltonian see the Appendix section (?). 

\subsection{Effective mass measurement}

We measure the effective mass of the Raman dressed atoms by adiabatically preparing our BECs in the lowest eigenstate and inducing dipole oscillations. The effective mass $m^{\star}$ of the dressed atoms  is related to the bare mass $m$ and the bare and dressed trapping frequencies $\omega$ and $\omega^{\star}$ by the ratio $m^{\star}/m=\sqrt{\omega^{\star}/\omega}$. We prepare our system in the  $\ket{m_F=0,\ k_x=0}$ and adiabatically turn on the Raman in $\sim10$ ms while also ramping the detuning to a non-zero value, around $0.5\Er$.  We then change the magnetic field away from resonance, shifting the minima in the ground state energy away from zero quasi-momentum. We then suddenly bring the field back to resonance which excites the dipole mode of our optical dipole trap. To measure the bare state frequency, we use the Raman beams to initially excite the dipole mode of the trap but subsequently turn them off ($\sim1\ms$) and let the BEC oscillate in the unmodified dipole potential. 

%Our system does not have the capability to dynamically change the laser frequency while maintaining phase stability, so unlike the pulsing experiments,%

For this set of measurements we modified our trapping frequencies to $(\omega_x, \omega_y, \omega_z)=2\pi(35.9, 32.5, xx) \Hz$  so that they were nominally symmetric along the $x-y$ plane. 

\subsection{Magnetic field stabilization}
We stabilized the magnetic field and measured fluctuations about the desired set point by applying a pair of microwave pulses to transfer a small fraction of atoms into $5^2{\rm S}_{1/2}$ $F=2$ state which we can non destructively image in-situ.

We first prepare our BEC in the $\ket{F=1, m_f=0}$ state and apply a $17.0556$ G bias filed along the z axis. We then apply a pair of $250\mu s$ microwave pulses close to $6.83 GHz$ that transfers about $10\%$ of the atoms into the $F=2$ manifold. The pulses were detuned by $\pm 2\kHz$ from the $\ket{F=1,m_F=0}\leftrightarrow\ket{F=2,m_F=1}$ transition and were spaced in time by $1/2\times 60 s$. We image the atoms transfered into $\ket{F=2,m_F=1}$ without disturbing atoms in $F=1$ since the imaging light is very detuned. The imbalance in the number of atoms transferred by each pulse gives us a $4\kHz$ wide error signal that we use both to feed forward our bias coils for active field stabilization, and also to keep track of the magnetic fields at each shot. We trigger our sequence to the line and both the microwave and Raman pulses are timed at integer periods of $60\Hz$ and performed at the zero-derivative point of the $60\Hz$ curve in order to minimize additional magnetic field fluctuations
