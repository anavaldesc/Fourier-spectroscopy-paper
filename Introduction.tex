\section{Introduction}
%Empezar con porque nos interesa soc (tal tez hacer enfasis en edge states, topological stuff, hofstadter). Describir el Hamiltoniano para un sistema de spin uno. Añadir la modulación  y explicar que modulación es equivalente a tres frequencias (citar spin one paper) espectralmente como en ciertos límites puedes tener acoplamiento cíclico. Por que quiero medir directamente las bandas? Mencionar que hay otras tecnicas como spin injection, pero que esta tecnica ofrece la ventaja de que no necesitas hardware adicional o "reservoir states" que solo ciertos elementos tienen. Motivar eso. De ahí voy a Fourier spectroscopy, funciona para un sistema conocido. Hamiltoniano modulato, comparar el gap de Molmer-Sorensen (acoplamiento entre -1 y +1) con la transición de dos fotones. Pensar en Mariposas de Hofstater, fin. 


Properties of materials deeply depend in their underlying band structure. Cold atoms offer engineering single particle dispersions that are analogues to condensed matter systems and new exotic materials, e.g. completely flat bands where interactions dominate the system, spin-orbit coupling. The ability to measure and control leads to new understanding and paves the way to access new phases of matter. 


Spin–orbit (SO) coupling is an essential mechanism for most
spintronics devices and leads to many fundamental phenomena
in condensed matter physics and atomic physics. For example,
SO coupling gives rise to the quantum spin Hall effect in
electronic condensed matter systems
quantum Hall effect, Floquet topological insulators, etc. 


The relation between the dynamics of a system and its energy spectrum is rooted in the heart of quantum mechanics.  This relation has been exploited before to study properties of both condensed mattter (una cita aqui) and cold atoms systems (otra cita) alike. Here, we propose a new Fourier spectroscopy technique to measure the dispersion relation of a system of spin-one, spin-orbit coupled ultra-cold atoms, which unlike previously studied techniques, relies only in the Hamiltonian evolution of the system and does not require any additional hardware or hyperfine states. This techniques works because when an atomic bare eigenstate is projected into a superposition of dressed states when a field is suddenly turned on, and will continue to evolve and Rabi oscillate in time, with spectral components related to the relative energies of the dressed states. 








We generate spin-orbit coupling in a spin-one atom using a pair of 'Raman' laser beams that change the spin state while imparting momentum to a spin-one atom via two photon Raman transitions.  

A uniform magnetic field $B\mathbf{\hat{e}}_z$ generates a linear Zeeman splitting of the energy levels $\hbar\omega_Z=g_F\mu_BB$, where $\mu_B$ is the Bohr magneton and $g_F$ is the Land\'e $g$ factor, and introduces a quadratic Zemen shift $\epsilon$ that shifts the energy of the $\ket{m_F=0}$ state with respect to the $\ket{m_F=\pm1}$ states. We couple these states using a pair of intersecting, cross polarized Raman beams with a wavelength $\lambda_R=790.024\nm$. We offset the frequency of the lasers by $\delta\omega=\omega_Z+\Delta_0$, where $\Delta_0$ is an experimentally controllable detuning from four photon resonance between the $\ket{m_F=-1}$ and $\ket{m_F=+1}$ states. 

The Raman field couples the state $\ket{m_F=0,\ q_x}$ to $\ket{m_F=-1,\ q_x+2k_L}$ and to $\ket{m_F=+1,\ q_x-2k_L}$, generating a spin change of $\delta m_F=\pm1$ and imparting a shift in momentum of $\pm 2k_L$, where $q_x$ denotes the quasimomentum. The geometry and wavelength of the Raman field determine the natural units of the system: the single photon recoil momentum $k_L=\frac{2\pi}{\lambda_R}\sin(\theta/2)$ and its associated recoil energy $E_L=\frac{\hbar^2k_L^2}{2m}$.

\begin{figure*}[bt]
	\begin{center}
		\includegraphics{figures/fig1_1v2.pdf}
		\caption
		{
			{\bf c)} SOC dispersion of a spin-one system with quadratic Zeeman shift of $9E_L$ and Raman coupling $\Omega_0=12E_L$, initially prepared at a momentum $k_x=2k_L$.
			{\bf d)} Probability amplitude of measuring the atoms in the state $\ket{m_F=-1,\ q_x=q_x+2k_L}$ (red), $\ket{m_F=0,\ q_x=q_x}$ (black), and $\ket{m_F=+1,\ q_x=q_x-2k_L}$ (blue) as a function of Raman pulsing time.
			{\bf e)} Fourier transform of the probability amplitude. The three peaks in the Fourier spectra correspond to the three relative energies in the SOC dispersion for the parameters described above.
			
		}
		\label{fig:Figure2}
	\end{center}
\end{figure*}

In a frame rotating at a frequency $\omega_Z$, and after a rotating wave approximation, the kinetic and atom-light contributions of the Hamiltonian along the recoil direction are
\begin{align}
\begin{split}
\hat{H_x} = &\frac{\hbar^2\hat{q}_x^2}{2m} + \alpha_0\hat{q}_x\hat{F}_z +4E_L\mathbb{I} + \frac{\Omega_R}{2}\hat{F}_x\\
& +(-\epsilon+4E_L)(\hat{F}_z^2-\mathbb{I}) +\Delta_0\hat{F}_z, 
\label{Eq:SOCone}
\end{split}]
\end{align}	

 where $\hat{F}_{x,y,z}$ are the spin-one matrices,  $\alpha_0=\frac{\hbar^2k_L}{m}$ is the spin-orbit coupling strength, $\Omega_R\propto E_A^{\star}E_B$ and the Raman coupling strength which is proportional to the field intensity. 
 
Figure 1a shows the typical band structure as a function of quasimomentum that results after diagonalizing the Hamiltonian $\hat{H}_x$ for a negative quadratic Zeeman shift $|\epsilon|>4E_L$. The ground state band is pushed down with respect to the higher excited bands, and it can be well described by a harmonic potential as there are no crossings with the higher bands. 

\subsection{Fourier spectroscopy of spin-orbit coupled atoms}		

 We can directly measure the dispersion relation of a system of spin-one, spin-orbit coupled bosonic atoms by studying the Hamiltonian evolution of the system, and unlike previous measurements of a SOC dispersion, our technique does not require any additional hardware. 

We start with bare atoms in the $\ket{m_F=0,\ k_x}$ state. When a Raman field is suddenly turned on, the initial state is projected into the Raman dressed state basis, and continues to evolve 
$\ket{m_F=0, q_x}\rightarrow \sum_{i=1}^3c_{i}e^{i\omega_it} \ket{\psi_i}$, where $\omega_i=E_i/\hbar$ are the angular frequencies associated to the dressed state energies and $\ket{\psi_i}$, the dressed eigenstates,  are linear combinations of $\ket{m_F=0,\ q_x=k_x}$ and $\ket{m_F=\pm1,\ q_x=k_x\mp2k_L}$. We then turn off the Raman field and image the atoms, which projects the system back into the bare basis $\ket{m_F=0,\ q_x=k_x}$, $\ket{m_F=\pm1,\ q_x=k_x\mp2k_L}$. The probability amplitude of measuring atoms in an $m_F$ state after evolving for a time $t$ oscillates at frequencies given by the difference in the dressed state energies $P_{m_F}(t)=\sum\limits_{i\neq j} 2c_{ij}\cos((\omega_i-\omega_j)t)$.

The Fourier spectroscopy technique relies in directly measuring the probability amplitude as a function of evolution time and extracting the relative energies of the system using a Fourier transform, as shown in figure \ref{fig:Figure2}.

 


The method described above so far only allows us to measure relative energies and we must add a known energy reference if we want to recover the dispersion relation. We can do so by measuring the effective mass $m^{\star} = \hbar^2[\frac{d^2E(k_x)}{dk_x}]^{-1}$ of the nearly quadratic lowest branch of the dispersion, and then shifting the measured frequencies accordingly. 

We can map the full spin and momentum dependent band structure of the spin-orbit coupled system by repeating this procedure for different initial momentum states, however, it may not be as straightforward to reliably prepare an arbitrary momentum state in the lab. The measurement however can be simplified by noticing that a non-moving atom cloud in the laboratory reference frame dressed by a field with non-zero detuning is equivalent to a moving cloud with a resonant field in a moving reference frame. This can be explicitly seen in the Hamiltonian \ref{Eq:SOCone} by looking at the detuning term $\Delta_0\hat{F}_z$ and the momentum term $\alpha_0\hat{q}_x\hat{F}_z$, which have the same effect in the relative energies. There is an additional Doppler shift associated with the transformation between reference frames, which gets canceled when we look at the energy differences. Therefore, for the purpose of our experiments, momentum and detuning are equivalent up to a numerical factor: $\Delta_0/E_R=4q_x/k_R$, and we can measure by preparing a zero momentum state and measuring the probability amplitude for different values of Raman detuning $\Delta_0$. 

%In order to maximize our signal to noise ratio (SNR) and minimize the required number of data points we use some fancy algorithm that I'm still not sure which one will work best. We also choose the spacing and the total number of pulses for each spectra so that the bandwith and resolution of the Fourier transform allow us to resolve the frequencies of interest. 

\subsection{Spectroscopy of a driven system}

A time evolution based spectroscopy is ideal to study the energy spectrum of more complex time dependent systems. A particularly interesting case is that of periodically driven systems [cita aqui], which can be described by effective coupling terms in the Hamiltonian that arise from averaging the fast dynamics of the system. The Fourier spectroscopy technique can be applied to such systems to measure the Floquet quasi-energy spectrum and also study the couplings within different Floquet manifolds. 

We will focus on the case of a spin-1 spin-orbit coupled system that is coupled by a multiple frequency Raman field, as shown in Fig 3b. The interference of the multiple frequencies leads to a periodic amplitude modulation in the field and an effective Floquet Hamiltonian that has tunable spin-orbit coupling. 

We add two sidebands to one Raman beam at angular frequencies $\omega=\omega_L+\omega_Z+\Delta_0 \pm \delta\omega$.  The Hamiltonian in Eq.\ref{Eq:SOCone} remains unchanged, except for the coupling strength that takes the form $	\Omega_R(t)=\Omega_0 + \Omega\cos(\delta\omega t)$. Our periodically driven system is well described by Floquet theory: the eigenstates are of the form $\ket{\Psi(t)}=\sum_{j}c_je^{i\epsilon_jt}\ket{u_j(t)}$ where $\ket{u(t)}$ are time-periodic states and the $\epsilon_j$ are the Floquet quasi-energies, wich are  $\epsilon_j,n=\epsilon_{j,m} + (n-m)2\pi/T$

\begin{figure*}
	\begin{center}
		\includegraphics{figures/fig3v1.pdf}
		\caption 
		{
			Time evolution of the BEC for Raman pulsing times between 5 and 10 $\mu s$, for different spin orbit coupling regimes:
			{\bf (i)} $\Omega_0=9.9 E_L$, $\Omega=04$,  $\Delta=5.8 E_L$, 
			{\bf (ii)} $\Omega_0=0$, $\Omega=8.6 E_L$,  $\Delta=-0.7 E_L$, and
			{\bf (iii)} $\Omega_0=1.5 E_L$, $\Omega=8.4 E_L$,  $\Delta=-4.7 E_L$
			
		}
		\label{fig:Figure3}
	\end{center}
\end{figure*}

The time evolution of the system after one driving cycle can be described in terms of an effective time-independent Hamiltonian $e^{iT\hat{H}_{eff}}$. If $\delta\omega \gg \epsilon$ and $\delta\omega \gg 4E_L$, this effective Floquet Hamiltonian retains the form of \ref{Eq:SOCone} with renormalized coefficients, and an additional term that couples the $m_f=-1$ and $m_f=+1$ states:

\begin{align}
	\begin{split}
		\hat{H} = &\frac{\hbar^2\hat{k}^2}{2m} + \alpha\hat{k}\hat{F}_z +4E_L\mathbb{I} + \frac{\Omega_0}{2}\hat{F}_x \\
		&+ \frac{\tilde{\Omega}}{2}\hat{F}_{xz} +(\tilde{\epsilon}+4E_L)(\hat{F}_z^2-\mathbb{I}) +\tilde{\Delta}\hat{F}_z, 
		\label{Eq:SOCeff}
	\end{split}
\end{align}	
%
with $\alpha= J_0(\Omega/2\delta\omega)\alpha_0$, $\tilde{\Omega}=\frac{1}{4}(\epsilon+4E_L) (J_0(\Omega/\delta\omega)-1)$, $\Delta=J_0(\Omega/2\delta\omega)\Delta_0$, and $\tilde{\epsilon}= \frac{1}{4}(4E_L-\epsilon) - 
\frac{1}{4}(4E_L + 3 \epsilon) J_0( \Omega/\delta\omega)$, and $J_0$ the zeroth order Bessel function of the first kind.

If the quadratic Zeeman shift and the amplitude modulation are large compared to the recoil energy $|\epsilon|, \delta\omega >4E_L$, the Hamiltonian described above can be approximated by an effective spin $1/2$, spin-orbit coupled system,with tunable spin-orbit coupling: the state $\ket{m_F=+1,\ q_x=k_x-2J_0(\Omega/2\delta\omega)k_R}$ is coupled to the $\ket{m_F=-1,\ q_x=k_x+2J_0(\Omega/2\delta\omega)k_R}$ with coupling strength  $\Omega'=\tilde{\Omega}+\hbar\Omega_0^2/2\tilde{\epsilon}$.

This model is good to describe the energies within one Floquet manifold, however, if the coupling strength is large compared to the quadratic Zeeman shift $\Omega_0,\ \Omega<\epsilon$, rotating wave type approximations break down, and the time evolution of the system becomes more complex. A signature of this is the appearance of higher frequency Fourier components in the probability amplitude, spaced by $2n\pi/T$ from the lowest 3 energy levels. 
 

%\begin{align}
%\begin{split}
%\hat{H}_{eff} = & \frac{\hbar^2}{2m}(\hat{k}+2\kr\hat{\sigma}_z)^2 + %\frac{\hbar\Omega'}{2}\hat{\sigma}_x  +\Delta\hat{\sigma}_z  
%\label{Eq:SOChalf}
%\end{split}
%\end{align}	



%Fig1b shows the high field dispersion relation, both for the modulated and unmodulated. The minima, originally locted at $\pm2 k_L$, are shifted and the size of the spin-orbit gap is changed for different choices of $\Omega_0$, $\Omega$, and $\delta\omega$.  







