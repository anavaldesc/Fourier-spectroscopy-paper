\section{Introduction}
%Empezar con porque nos interesa soc (tal tez hacer enfasis en edge states, topological stuff, hofstadter). Describir el Hamiltoniano para un sistema de spin uno. Añadir la modulación  y explicar que modulación es equivalente a tres frequencias (citar spin one paper) espectralmente como en ciertos límites puedes tener acoplamiento cíclico. Por que quiero medir directamente las bandas? Mencionar que hay otras tecnicas como spin injection, pero que esta tecnica ofrece la ventaja de que no necesitas hardware adicional o "reservoir states" que solo ciertos elementos tienen. Motivar eso. De ahí voy a Fourier spectroscopy, funciona para un sistema conocido. Hamiltoniano modulato, comparar el gap de Molmer-Sorensen (acoplamiento entre -1 y +1) con la transición de dos fotones. Pensar en Mariposas de Hofstater, fin. 


%Point to sell: modulations are cool and deep in the heart of atomic physics. Maybe talk about pulsed optical lattices? Modulated Raman stuff from the group with cool frequency ramps? Read Dalibards paper on modulated systems for inspiration, Read ians paper on detecting topological matter using cold atoms for more inspiration.

The the relation between the dynamics/time evolution of a system is rooted in the heart of quantum mechanics. 

Does driving strength exceed transition frequency? No, but it does exceed the quadratic zeeman shift. 

From the floquet paper: The observed  system dynamics is very well described in terms of
quasienergies and quasienergy states, as predicted by
Floquet theory. In particular, we observe several frequency
components in the dynamics, in very good agreement with
theory

Think about the coupling strength vs quadratic zeeman shift in terms of excited floquet states. 

Think about fourier spectroscopy for Bloch bands.

In analogy to solid state stuff we introduce a new Fourier based spectroscopy technique that we use to measure the spin dependent energy-momentum dispersion bands of the system. With the addition of a one dimensional optical lattice, this work opens the ground for measurements of the Hofstadter butterfly spectrum. 


Previous studies have shown that strong modulation in the Raman coupling strength for an effective spin $1/2$ system leads to tunable spin-orbit coupling strength. In addition to this, the use of two Raman coupling frequencies in a spin one system which is equivalent to a single frequency amplitude modulated coupling leads to new magnetic phases (cite spin one papar). Here we extend the study of the effects of amplitude modulation of the Raman coupling strength/multiple frequency couplings. We will show that for a spin one system, we can independently  tune the spin orbit-coupling gap and strength, and we can additionally engineer a cyclic coupling between the three $m_F$ magnetic sub-levels. 



Fixes:
\begin{itemize}
	\item Write frequencies as $\omega_L$ and $\omega_L +\Delta\omega \pm\delta\omega$. Is there too many $\delta$ symbols, confusing?
	\item Call $J_0$ the zeroth order Bessel functions of the first kind.
	\item don't say modulation or multiple frequencies, just say we amplitude modulate by using 
	\item no effective model, say the floquet Hamiltonian takes de form
	\item mention 2 paths
\end{itemize}


effective Hamiltonian that captures the essential characteristics
of the modulated system. This strategy exploits the
fact that modulation schemes can be tailored in such a way
that effective Hamiltonians reproduce the Hamiltonians
of interesting static systems.

Corrections of the effective Hamiltonian are of the order $1/\deltaº\omega$.

Order of ideas here:
First spectroscopy. Describe for a single frequency spin one SOC Hamiltonian. Describe the periodic drive case, brief intro to Floquet theory.  
 
