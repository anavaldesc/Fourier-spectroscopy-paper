\section{Introduction}

Properties of materials deeply depend in their underlying band structure. Cold atoms systems offer an exciting possibility to engineer single particle dispersions that are analogues to condensed matter systems and new exotic materials, e.g. completely flat bands where interactions dominate the system, spin-orbit systems, systems with non-trivial topology \cite{lindner_floquet_2011,radic_strong_2015}.... The ability to measure the engineered dispersions leads to a deeper understanding and paves the way to accessing new phases of matter. 
 
The relation between the energy spectrum of a system and its dynamics is rooted in the heart of quantum mechanics.  This relation has been exploited before to study spectral properties of both condensed matter [need a good cite here] and cold atoms systems \cite{yoshimura_diabatic-ramping_2014,wang_atom-interferometric_2015} alike. We propose a new Fourier spectroscopy technique to measure the dispersion relation of spin-orbit coupled ultra-cold atoms, which unlike previously studied techniques \cite{cheuk_spin-injection_2012}, relies only in the time evolution of the Hamiltonian, as the bare atomic eigenstates are projected into a superposition of dressed states when a field is suddenly turned on, and proceed to evolve under that field. The spectral components of the Fourier transform then correspond to the relative energies of the dressed eigenstates. % that evolves in time with spectral components related to the relative energies of the dressed states. Our technique can be particularly useful to study driven systems, where it is hard to predict how the different Floquet manifolds of the system couple. 

We engineered a dispersion relation that has equal contributions of Rashba and Dresselhaus spin-orbit coupling (SOC) by coupling the different spin projections of a spin-one system laser fields \cite{dalibard_textitcolloquium_2011}. We study the case of SOC driven by a single frequency, which is equivalent to a time independent problem in a rotating frame, and SOC driven by multiple frequencies, which is equivalent to periodically driven system but can be described by an effective time independent model with tunable SOC strength \cite{jimenez-garcia_tunable_2015}.

We generate the spin-orbit coupling using a pair of 'Raman' laser beams that change the spin state while imparting a spin-dependent  momentum to a spin-one atom via two photon Raman transitions. Our system consists of ultra-cold $^{87}\mathrm{Rb}$ atoms, and we use the magnetic sub-levels of the $F=1$ ground state hyperfine manifold  as different spin projections \cite{lan_raman-dressed_2014,campbell_itinerant_2015}.  A uniform magnetic field $B\mathbf{\hat{e}}_z$ generates a linear Zeeman splitting of the energy levels $\hbar\omega_Z=g_F\mu_BB$, where $\mu_B$ is the Bohr magneton and $g_F$ is the Land\'e $g$ factor, and introduces a quadratic Zemen shift $\epsilon$ that shifts the energy of the $\ket{m_F=0}$ state with respect to the $\ket{m_F=\pm1}$ states. We couple the three states using a pair of intersecting, cross polarized Raman beams with angular frequency $\omega_A$ and $\omega_B=\omega_A+\omega_Z+\Delta_0$, where $\Delta_0$ is an experimentally controllable detuning from four photon resonance between the $\ket{m_F=-1}$ and $\ket{m_F=+1}$ states. The Raman field couples the state $\ket{m_F=0,\ q_x}$ to $\ket{m_F=-1,\ q_x+2k_L}$ and to $\ket{m_F=+1,\ q_x-2k_L}$, generating a spin change of $\delta m_F=\pm1$ and imparting a momentum of $\pm 2k_L$, where $q_x$ denotes the quasimomentum. The geometry and wavelength of the Raman field determine the natural units of the system: the single photon recoil momentum $k_L=\frac{2\pi}{\lambda_R}\sin(\theta/2)$, with $\lambda_R$ the wavelength of the Raman laser and $\theta$ the relative angle between the Raman beams, and its associated recoil energy $E_L=\frac{\hbar^2k_L^2}{2m}$.

\begin{figure*}[bt]
	\begin{center}
		\includegraphics{figures/fig1_1v2.pdf}
		\caption
		{
			{\bf c)} SOC dispersion of a spin-one system with quadratic Zeeman shift of $9E_L$ and Raman coupling $\Omega_0=12E_L$, initially prepared at a momentum $k_x=2k_L$.
			{\bf d)} Probability amplitude of measuring the atoms in the state $\ket{m_F=-1,\ q_x=q_x+2k_L}$ (red), $\ket{m_F=0,\ q_x=q_x}$ (black), and $\ket{m_F=+1,\ q_x=q_x-2k_L}$ (blue) as a function of Raman pulsing time.
			{\bf e)} Fourier transform of the probability amplitude. The three peaks in the Fourier spectra correspond to the three relative energies in the SOC dispersion for the parameters described above.
			
		}
		\label{fig:Figure1}
	\end{center}
\end{figure*}

After rotating to a frame at frequency $\omega_Z$ and a rotating wave approximation, the kinetic and atom-light contributions of the Hamiltonian along the recoil direction are
\begin{align}
\begin{split}
\hat{H_x} = &\frac{\hbar^2\hat{q}_x^2}{2m} + \alpha_0\hat{q}_x\hat{F}_z +4E_L\mathbb{I} + \frac{\Omega_R}{2}\hat{F}_x\\
& +(-\epsilon+4E_L)(\hat{F}_z^2-\mathbb{I}) +\Delta_0\hat{F}_z, 
\label{Eq:SOCone}
\end{split}]
\end{align}	

 where $\hat{F}_{x,y,z}$ are the spin-one matrices,  $\alpha_0=\frac{\hbar^2k_L}{m}$ is the SOC strength, and $\Omega_R\propto E_A^{\star}E_B$ is the Raman coupling strength which is proportional to the field intensity. We have also defined the direction of the $\mathbf{e}_x$ axis to be parallel to $k_L$.
 
Figure \ref{fig:Figure1}c shows a typical band structure of a SOC system as a function of quasimomentum that results of diagonalizing the Hamiltonian $\hat{H}_x$ for a negative quadratic Zeeman shift $|\epsilon|>4E_L$. The ground state band is pushed to lower energy with respect to the higher excited bands, and it can be well described by a harmonic potential as there are no crossings with the higher bands. 

\subsection{Fourier spectroscopy of spin-orbit coupled atoms}		

We can directly measure the dispersion relation of a system of spin-one, spin-orbit coupled atoms by studying the time evolution of the Hamiltonian.

We start with bare atoms in the $\ket{m_F=0,\ k_x}$ state. When a Raman field is suddenly turned on, the initial state is projected into the Raman dressed state basis, and continues to evolve 
$\ket{m_F=0, q_x}\rightarrow \sum_{i=1}^3c_{i}e^{i\omega_it} \ket{\psi_i}$, where $\omega_i=E_i/\hbar$ are the angular frequencies associated to the dressed state energies and $\ket{\psi_i}$, the dressed eigenstates,  are linear combinations of $\ket{m_F=0,\ q_x=k_x}$ and $\ket{m_F=\pm1,\ q_x=k_x\mp2k_L}$. We then suddenly turn off the Raman field and image the atoms, which projects the system back into the bare basis $\ket{m_F=0,\ q_x=k_x}$, $\ket{m_F=\pm1,\ q_x=k_x\mp2k_L}$. The probability amplitude of measuring atoms in an $m_F$ state after evolving for a time $t$ oscillates at frequencies given by the difference in the dressed state energies $P_{m_F}(t)=\sum\limits_{i\neq j} 2c_{ij}\cos((\omega_i-\omega_j)t)$.

The Fourier spectroscopy technique relies in directly measuring the probability amplitude as a function of time and extracting the relative energies of the system using a Fourier transform, as shown in figure \ref{fig:Figure2}d, e.

The method described above so far only allows us to measure relative energie. To recover the dispersion relations we must add a known energy reference by measuring the effective mass $m^{\star} = \hbar^2[\frac{d^2E(k_x)}{dk_x}]^{-1}$ of the nearly quadratic lowest branch of the dispersion, and then shifting the measured frequencies accordingly. 

We can map the full spin and momentum dependent band structure of the SOC system by repeating this procedure for different initial momentum states, however, non-moving atom cloud in the laboratory reference frame dressed by a laser field with non-zero detuning is equivalent to a moving cloud with a resonant field in the cloud reference frame as can be explicitly seen in the Eq. \ref{Eq:SOCone}. The detuning term $\Delta_0\hat{F}_z$ and the momentum term $\alpha_0\hat{q}_x\hat{F}_z$, have the same effect in the relative energies. All the energies are shifted when the  associated with the transformation between reference frames, which gets canceled when we look at the energy differences. Therefore, for the purpose of our experiments, momentum and detuning are equivalent up to a numerical factor: $\Delta_0/E_R=4q_x/k_R$.% and we can measure by preparing a zero momentum state and measuring the probability amplitude for different values of Raman detuning $\Delta_0$. 

\subsection{Spectroscopy of periodically driven SOC system}

A time-domain based spectroscopy is ideally suited to study the energy spectrum of more complex time dependent systems. A particularly interesting case is that of periodically driven systems \cite{jimenez-garcia_tunable_2015,eckardt_superfluid-insulator_2005,goldman_periodically_2014}, which are well suited for engineering and tuning Hamiltonians, as they can be described by effective coupling terms that arise from averaging the fast dynamics of the system. %The Fourier spectroscopy technique can be applied to such systems to measure the Floquet quasi-energy spectrum and also study the couplings within different Floquet manifolds. 

We will study a spin-1 SOC system that is coupled by a multiple frequency Raman field, as shown in Fig \ref{fig:Figure2}. The interference of the multiple frequencies leads to a periodic amplitude modulation in the field and an effective Floquet Hamiltonian with a tunable SOC strength term \cite{jimenez-garcia_tunable_2015}.

We add two sidebands to one Raman beam at angular frequencies $\omega=\omega_L+\omega_Z+\Delta_0 \pm \delta\omega$.  The Hamiltonian in Eq.\ref{Eq:SOCone} remains unchanged, except for the coupling strength that takes the form $	\Omega_R(t)=\Omega_0 + \Omega\cos(\delta\omega t)$. Our periodically driven system is well described by Floquet theory. The eigenstates are of the form $\ket{\Psi_{\epsilon_m}(t)}=\sum\limits_{l=-\infty}^{+\infty}e^{-i(\epsilon_m-l\delta\omega) t}\ket{\psi_{\epsilon_m,l}(t)}$ where $\ket{\psi_{\epsilon_m,l}(t)}$ are time-periodic states and $\epsilon_m+l\delta\omega$ are the quasi-energies.

\begin{figure*}
	\begin{center}
		\includegraphics{figures/fig3v1.pdf}
		\caption 
		{
		We use a multiple frequency Raman beam with up to 3 frequencies that gives rise to a system with tunable spin-orbit coupling. 
			{\bf (a)} Frequency components in each Raman beam
			{\bf (b)} The combination of a blue and red sideband are at four photon resonance between the $\ket{m_F=+1}$ and $\ket{m_F=-1}$ state and there are two possible paths to couple these states. 
			
		}
		\label{fig:Figure2}
	\end{center}
\end{figure*}

Rather than solving the full Floquet problem, we calculate the propagator for one driving period $T=2\pi/\delta\omega$ and define an effective, time-independent, Hamiltonian  $e^{iT\hat{H}_{eff}}$, with $\delta\omega \gg 4E_L$. This effective Floquet Hamiltonian retains the form of Eq.\ref{Eq:SOCone} with renormalized coefficients, and an additional term that couples the $m_f=-1$ and $m_f=+1$ states:

\begin{align}
	\begin{split}
		\hat{H} = &\frac{\hbar^2\hat{k}^2}{2m} + \alpha\hat{k}\hat{F}_z +4E_L\mathbb{I} + \frac{\Omega_0}{2}\hat{F}_x \\
		&+ \frac{\tilde{\Omega}}{2}\hat{F}_{xz} +(\tilde{\epsilon}+4E_L)(\hat{F}_z^2-\mathbb{I}) +\tilde{\Delta}\hat{F}_z, 
		\label{Eq:SOCeff}
	\end{split}
\end{align}	
%
where $\alpha= J_0(\Omega/2\delta\omega)\alpha_0$, $\tilde{\Omega}=\frac{1}{4}(\epsilon+4E_L) (J_0(\Omega/\delta\omega)-1)$, $\Delta=J_0(\Omega/2\delta\omega)\Delta_0$, and $\tilde{\epsilon}= \frac{1}{4}(4E_L-\epsilon) - 
\frac{1}{4}(4E_L + 3 \epsilon) J_0( \Omega/\delta\omega)$, and $J_0$ the zero order Bessel function of the first kind.

When the quadratic Zeeman shift and the driving frequency are large compared to the recoil energy $|\epsilon|, \delta\omega >4E_L$, the Hamiltonian in Eq. \ref{Eq:SOCeff} can be approximated by an effective spin $1/2$, SOC system,with tunable spin-orbit coupling: the state $\ket{m_F=+1,\ q_x=k_x-2J_0(\Omega/2\delta\omega)k_R}$ is coupled to the $\ket{m_F=-1,\ q_x=k_x+2J_0(\Omega/2\delta\omega)k_R}$ with coupling strength  $\Omega'=\tilde{\Omega}+\hbar\Omega_0^2/2\tilde{\epsilon}$ \cite{l._j._leblanc_direct_2013}. In order to get a full control over the SOC gap and the SOC strength, we need to have control over the parameters $\delta\omega$, $\Omega$ and $\Omega_0$.

The effective Hamiltonian correctly captures the energies within one Floquet manifold, however, if the coupling strength becomes comparable to the driving frequency $\Omega_0,\ \Omega\geq\delta\omega$, the RWA breaks down and the counter-rotating terms can not be ignored. The signature of this is the appearance of higher frequency Fourier components in the probability amplitude, spaced by $\delta\omega$ from the lowest three energy levels. We expect the band structure from this type of system to be more complex, but easily interpreted by understanding the periodically repeating structure of the Floquet quasi-energies. 
 

%\begin{align}
%\begin{split}
%\hat{H}_{eff} = & \frac{\hbar^2}{2m}(\hat{k}+2\kr\hat{\sigma}_z)^2 + %\frac{\hbar\Omega'}{2}\hat{\sigma}_x  +\Delta\hat{\sigma}_z  
%\label{Eq:SOChalf}
%\end{split}
%\end{align}	



%Fig1b shows the high field dispersion relation, both for the modulated and unmodulated. The minima, originally locted at $\pm2 k_L$, are shifted and the size of the spin-orbit gap is changed for different choices of $\Omega_0$, $\Omega$, and $\delta\omega$.  







