\documentclass{iopart}
%[amsmath,amssymb,floatfix]

\usepackage{iopams}
\expandafter\let\csname equation*\endcsname\relax
\expandafter\let\csname endequation*\endcsname\relax
\usepackage{amsmath}


\usepackage{graphicx}       % Include figure files
\newcommand{\documentpath}{./Figures}
%%
%% macros.tex
%%

\newcommand{\onlinecite}[1]{\nocite{#1}\citenum{#1}}
% Changes
\def\cs{{\ {\clubsuit}}}

% Length

\def\nm{{\ {\rm nm}}}						% nm
\def\mm{{\ {\rm mm}}}						% mm
\def\cm{{\ {\rm cm}}}						% cm
\def\micron{{\ \mu{\rm m}}}					% Microns
\def\angstrom{{\ \mbox{{\rm \AA}}}}			% Angstroms

% Electrons
\def\tesla{{\ {\rm T}}}						% tesla
\def\nohm{{\ {\rm n}\Omega}}				% nohm
\def\uohm{{\ \mu\Omega}}					% uohm
\def\mohm{{\ {\rm m}\Omega}}				% mohm
\def\ohm{{\ \Omega}}						% ohm
\def\kohm{{\ {\rm k}\Omega}}				% Kohm
\def\Mohm{{\ {\rm M}\Omega}}				% Mohm
\def\conductance#1{{\times{#1}\ \Omega^{-1}}} % Mhos

\def\density#1{{\times{#1}\ {\rm cm}^{-2}}}				% Density
\def\mobility#1{{\times{#1}\ {\rm cm}^{2}/{\rm V s}}} 	% Mobility
\def\microvolt{{\ \mu{\rm V}}}				% Microvolts
\def\volt{{\ {\rm V}}}						% volts

% Current
\def\pA{{\ {\rm pA}}}						% pA
\def\nA{{\ {\rm nA}}}						% nA
\def\uA{{\ \mu{\rm A}}}						% uA
\def\mA{{\ {\rm mA}}}						% mA
\def\amp{{\ {\rm A}}}						% A
\def\kA{{\ {\rm kA}}}						% kA
\def\MA{{\ {\rm MA}}}						% MA
\def\GA{{\ {\rm GA}}}						% GA
\def\TA{{\ {\rm TA}}}						% TA

% Vots
\def\volt{{\ {\rm V}}}						% V
\def\mV{{\ {\rm mV}}}						% mV
\def\uV{{\ \mu{\rm V}}}					% uV
\def\nV{{\ {\rm nV}}}						% nV
\def\pV{{\ {\rm pV}}}						% pV
\def\fV{{\ {\rm fV}}}						% fV

% Energy
\def\eV{{\ {\rm eV}}}						% eV
\def\meV{{\ {\rm meV}}}						% meV
\def\ueV{{\ \mu{\rm eV}}}					% ueV
\def\neV{{\ {\rm neV}}}						% neV

% Frequency
\def\uHz{{\ \mu{\rm Hz}}}					% uHz
\def\mHz{{\ {\rm mHz}}}						% mHz
\def\Hz{{\ {\rm Hz}}}						% Hz
\def\kHz{{\ {\rm kHz}}}						% kHz
\def\MHz{{\ {\rm MHz}}}						% MHz
\def\GHz{{\ {\rm GHz}}}						% GHz
\def\THz{{\ {\rm THz}}}						% THz

% Time
\def\fs{{\ {\rm fs}}}						% fs
\def\ps{{\ {\rm ps}}}						% ps
\def\ns{{\ {\rm ns}}}						% ns
\def\us{{\ \mu{\rm s}}}						% us
\def\ms{{\ {\rm ms}}}						% ms
\def\second{{\ {\rm s}}}					% s

% Temperature
\def\kelvin{{\ {\rm K}}}					% K
\def\mK{{\ {\rm mK}}}						% mK
\def\uK{{\ \mu{\rm K}}}						% uK
\def\nK{{\ {\rm nK}}}						% nK

% mass
\def\kg{{\ {\rm kg}}}					% kg
\def\gram{{\ {\rm g}}}					% g
\def\mg{{\ {\rm mg}}}						% mg
\def\ug{{\ \mu{\rm g}}}						% ug
\def\ng{{\ {\rm ng}}}						% ng


% Magnetic Field
\def\gauss{{\ {\rm G}}}					% T
\def\tesla{{\ {\rm T}}}					% T
\def\mT{{\ {\rm mT}}}						% mT
\def\uT{{\ \mu{\rm T}}}						% uT
\def\nT{{\ {\rm nT}}}						% nT

% AMO abbriviations
\def\Er{{{E_R}}}							% Er
\def\kr{{{k_R}}}							% kr
\def\Rb87{^{87}\rm{Rb}}					% Rb 87
\def\UoverTC{(U/t)_{\rm c}}					% t/U_c

\def\nbar{\left<{\hat n}_k\right>}					% average number
\def\nbarexpt{\left<n(k_x,k_y)\right>}				% average number

% Specific Symbols
\def\DeltaSAS{{\Delta_{{\rm SAS}}}}			% DeltaSAS
\def\HeFour{{^4{\rm He}}}					% Helium 4
\def\HeThree{{^3{\rm He}}}					% Helium 3
\def\lb{{l_B}}								% Magnetic Length
\def\DoverL{{d/\lb}}						% d/l
\def\DoverLcrit{{\DoverL_{\rm crit}}}		% d/l_crit
\def\Bpar{{B_{\|}}}							% B Parallel
\def\Bperp{{B_{\perp}}}						% B Perpindicular
\def\AlGaAs#1#2{{{\rm Al}_{#1}{\rm Ga}_{#2}{\rm As}}} % Al_xGa_{1-x}As
\def\Schrodinger{{Schr\"odinger\ }}

% Basic mathematical symbols
\def\ex{{\mathbf e}_x}                            % e_x
\def\ey{{\mathbf e}_y}                            % e_y
\def\ez{{\mathbf e}_z}                            % e_z
\def\epos{{\mathbf e}_+}                            % e_z
\def\eneg{{\mathbf e}_-}                            % e_z
\def\shorttimes{\!\times\!}                            % e_z
\def\shorteq{\! = \!}                            % e_z
\DeclareMathAlphabet\mathbfcal{OMS}{cmsy}{b}{n}


% Commands for bra-ket notation
\newcommand{\bra}[1]{\langle #1|}
\newcommand{\ket}[1]{|#1\rangle}
\newcommand{\braket}[2]{\langle #1|#2\rangle}
\newcommand{\checkhat}[1]{\overset{\times}{#1}}
%\newcommand{\checkhat}[1]{\check{\hat #1}}


\begin{document}
		
\title{Fourier spectroscopy of a spin-orbit coupled Bose gas}
	
\author{Ana Vald\'es-Curiel, Dimitri Trypogeorgos, Erin E. Marshall, Ian B. Spielman}
\address{Joint Quantum Institute, University of Maryland and National Institute of Standards and Technology, College Park, Maryland, 20742, USA}
\date{\today}

\begin{abstract}
	%Here we look at soc-ed bose gas with modulation. Technique generalizes the proposal of (Karina) and opens the door to generating cyclic couplings between $m_f=+1$ and $m_f=+1$ we propose a new method of probing the energy-momentum dispersion of a spin-orbit coupled quantum gas
	
%	The coupling between an electron's spin and motion lies withing the heart cool things in condensed matter systems. The high degree of control in cold atoms systems makes them ideal candidates for studying new phases of matter and simulating quantum systems.
	 We propose a time domain technique to measure the band structure of a spin-1 spin-orbit coupled Bose-Einstein condensate that relies on the Hamiltonian evolution of the system. We drive transitions at different values of detuning from Raman resonance and extract the Fourier components of the time dependent evolution to reconstruct the spin and momentum dependent energy spectrum. We add a periodic modulation to one Raman field which results in a tunable spin-orbit coupling dispersion and a spectrum of Floquet quasi-energies that we can directly measure, showing the robustness of our technique.  
\end{abstract}

\maketitle
%\tableofcontents

\section{Introduction}
%Empezar con porque nos interesa soc (tal tez hacer enfasis en edge states, topological stuff, hofstadter). Describir el Hamiltoniano para un sistema de spin uno. Añadir la modulación  y explicar que modulación es equivalente a tres frequencias (citar spin one paper) espectralmente como en ciertos límites puedes tener acoplamiento cíclico. Por que quiero medir directamente las bandas? Mencionar que hay otras tecnicas como spin injection, pero que esta tecnica ofrece la ventaja de que no necesitas hardware adicional o "reservoir states" que solo ciertos elementos tienen. Motivar eso. De ahí voy a Fourier spectroscopy, funciona para un sistema conocido. Hamiltoniano modulato, comparar el gap de Molmer-Sorensen (acoplamiento entre -1 y +1) con la transición de dos fotones. Pensar en Mariposas de Hofstater, fin. 


%Point to sell: modulations are cool and deep in the heart of atomic physics. Maybe talk about pulsed optical lattices? Modulated Raman stuff from the group with cool frequency ramps? Read Dalibards paper on modulated systems for inspiration, Read ians paper on detecting topological matter using cold atoms for more inspiration.


Spin–orbit (SO) coupling is an essential mechanism for most
spintronics devices and leads to many fundamental phenomena
in condensed matter physics and atomic physics. For example,
SO coupling gives rise to the quantum spin Hall effect in
electronic condensed matter systems

Start with corny paragraph, about soc and how cold atoms are awesome.
Describe SOC at high field and at four photon resonance. Describe Fourier spectroscopy. Describe modulated Raman. 

The relation between the dynamics of a system and its energy spectrum is rooted in the heart of quantum mechanics.  This relation has been exploited before to study properties of both condensed mattter (una cita aqui) and cold atoms systems (otra cita) alike.



The  properties  of  electronic  materials  are  deeply
entwined with their band structure
—
or, more generally,
their single-particle spectrum
—
which gives rise to con-
ductors, semiconductors, conventional insulators, and now
topological insulators
[1]
. Understanding and controlling
band structure in new ways, therefore, allows access to new
phenomena. Spin-orbit coupling (SOC) plays a fundamen-
tal role in most topological materials, linking the spin and
the momentum of quantum particles. The introduction of
time-periodic perturbations to topologically trivial systems
(quantum wells, solid-state materials, and ultracold atoms)
can drive phase transitions to new
“
Floquet topological
phases
”
[2,3]
. For example, Floquet topological insulators
arise from topologically trivial materials with spin-orbit
coupling through time-periodic perturbations
[2]
.


In order to explicitly measure the  energy-momentum dispersion relation, we will use a Fourier based spectroscopy technique, which relies on the time evolution of an atomic state after a dressing field is suddenly turned on, and the initially bare states become superpositions of dressed states undergoing Rabi oscillations in time  with spectral components related to the relative energies of the dressed states. 


We generate spin-orbit coupling in a spin-one atom using a pair of 'Raman' laser beams that change the spin state while imparting momentum to a spin-one atom via two photon Raman transitions.  

A uniform magnetic field $B\mathbf{\hat{e}}_z$ generates a linear Zeeman splitting of the energy levels $\hbar\omega_Z=g_F\mu_BB$, where $\mu_B$ is the Bohr magneton and $g_F$ is the Land\'e $g$ factor, and introduces a quadratic Zemen shift $\epsilon$ that shifts the energy of the $\ket{m_F=0}$ state with respect to the $\ket{m_F=\pm1}$ states. We couple these states using a pair of intersecting, cross polarized Raman beams with a wavelength $\lambda_R=790.024\nm$. We offset the frequency of the lasers by $\delta\omega=\omega_Z+\Delta_0$, where $\Delta_0$ is an experimentally controllable detuning from four photon resonance between the $\ket{m_F=-1}$ and $\ket{m_F=+1}$ states. 

The Raman field couples the state $\ket{m_F=0,\ q_x}$ to $\ket{m_F=-1,\ q_x+2k_L}$ and to $\ket{m_F=+1,\ q_x-2k_L}$, generating a spin change of $\delta m_F=\pm1$ and imparting a shift in momentum of $\pm 2k_L$, where $q_x$ denotes the quasimomentum. The geometry and wavelength of the Raman field determine the natural units of the system: the single photon recoil momentum $k_L=\frac{2\pi}{\lambda_R}\sin(\theta/2)$ and its associated recoil energy $E_L=\frac{\hbar^2k_L^2}{2m}$.

\begin{figure*}[bt]
	\begin{center}
		\includegraphics{figures/fig1v0.pdf}
		\caption
		{
			{\bf a)} SOC dispersion of a spin-one system with quadratic Zeeman shift of $9E_L$ and Raman coupling $\Omega_0=12E_L$, initially prepared at a momentum $k_x=2k_L$.
			{\bf b)} Probability amplitude of measuring the atoms in the state $\ket{m_F=-1,\ q_x=q_x+2k_L}$ (red), $\ket{m_F=0,\ q_x=q_x}$ (black), and $\ket{m_F=+1,\ q_x=q_x-2k_L}$ (blue) as a function of Raman pulsing time.
			{\bf c)} Fourier transform of the probability amplitude. The three peaks in the Fourier spectra correspond to the three relative energies in the SOC dispersion for the parameters described above.
			
		}
		\label{fig:Figure1}
	\end{center}
\end{figure*}

In a frame rotating at a frequency $\omega_Z$, and after a rotating wave approximation, the kinetic and atom-light contributions of the Hamiltonian along the recoil direction are
\begin{align}
\begin{split}
\hat{H_x} = &\frac{\hbar^2\hat{q}_x^2}{2m} + \alpha_0\hat{q}_x\hat{F}_z +4E_L\mathbb{I} + \frac{\Omega_R}{2}\hat{F}_x\\
& +(-\epsilon+4E_L)(\hat{F}_z^2-\mathbb{I}) +\Delta_0\hat{F}_z, 
\label{Eq:SOCone}
\end{split}]
\end{align}	

 where $\hat{F}_{x,y,z}$ are the spin-one matrices,  $\alpha_0=\frac{\hbar^2k_L}{m}$ is the spin-orbit coupling strength, $\Omega_R\propto E_A^{\star}E_B$ and the Raman coupling strength which is proportional to the field intensity. 
 
Figure 1a shows the typical band structure as a function of quasimomentum that results after diagonalizing the Hamiltonian $\hat{H}_x$ for a negative quadratic Zeeman shift $|\epsilon|>4E_L$. The ground state band is pushed down with respect to the higher excited bands, and it can be well described by a harmonic potential as there are no crossings with the higher bands. 

\subsection{Fourier spectroscopy of spin-orbit coupled atoms}		

 We can directly measure the dispersion relation of a system of spin-one, spin-orbit coupled bosonic atoms by studying the Hamiltonian evolution of the system, and unlike previous measurements of a SOC dispersion, our technique does not require any additional hardware. 

We start with bare atoms in the $\ket{m_F=0,\ k_x}$ state. When a Raman field is suddenly turned on, the initial state is projected into the Raman dressed state basis, and continues to evolve 
$\ket{m_F=0, q_x}\rightarrow \sum_{i=1}^3c_{i}e^{i\omega_it} \ket{\psi_i}$, where $\omega_i=E_i/\hbar$ are the angular frequencies associated to the dressed state energies and $\ket{\psi_i}$, the dressed eigenstates,  are linear combinations of $\ket{m_F=0,\ q_x=k_x}$ and $\ket{m_F=\pm1,\ q_x=k_x\mp2k_L}$. We then turn off the Raman field and image the atoms, which projects the system back into the bare basis $\ket{m_F=0,\ q_x=k_x}$, $\ket{m_F=\pm1,\ q_x=k_x\mp2k_L}$. The probability amplitude of measuring atoms in an $m_F$ state after evolving for a time $t$ oscillates at frequencies given by the difference in the dressed state energies $P_{m_F}(t)=\sum\limits_{i\neq j} 2c_{ij}\cos((\omega_i-\omega_j)t)$.

The Fourier spectroscopy technique relies in directly measuring the probability amplitude as a function of evolution time and extracting the relative energies of the system using a Fourier transform, as shown in figure \ref{fig:Figure2}.

 
\begin{figure*}[bt]
	\begin{center}
		\includegraphics{figures/fourier2.pdf}
		\caption
		{
			{\bf a)} SOC dispersion of a spin-one system with quadratic Zeeman shift of $9E_L$ and Raman coupling $\Omega_0=12E_L$, initially prepared at a momentum $k_x=2k_L$.
			{\bf b)} Probability amplitude of measuring the atoms in the state $\ket{m_F=-1,\ q_x=q_x+2k_L}$ (red), $\ket{m_F=0,\ q_x=q_x}$ (black), and $\ket{m_F=+1,\ q_x=q_x-2k_L}$ (blue) as a function of Raman pulsing time.
			{\bf c)} Fourier transform of the probability amplitude. The three peaks in the Fourier spectra correspond to the three relative energies in the SOC dispersion for the parameters described above.
			
		}
		\label{fig:Figure2}
	\end{center}
\end{figure*}

The method described above so far only allows us to measure relative energies and we must add a known energy reference if we want to recover the dispersion relation. We can do so by measuring the effective mass $m^{\star} = \hbar^2[\frac{d^2E(k_x)}{dk_x}]^{-1}$ of the nearly quadratic lowest branch of the dispersion, and then shifting the measured frequencies accordingly. 

We can map the full spin and momentum dependent band structure of the spin-orbit coupled system by repeating this procedure for different initial momentum states, however, it may not be as straightforward to reliably prepare an arbitrary momentum state in the lab. The measurement however can be simplified by noticing that a non-moving atom cloud in the laboratory reference frame dressed by a field with non-zero detuning is equivalent to a moving cloud with a resonant field in a moving reference frame. This can be explicitly seen in the Hamiltonian \ref{Eq:SOCone} by looking at the detuning term $\Delta_0\hat{F}_z$ and the momentum term $\alpha_0\hat{q}_x\hat{F}_z$, which have the same effect in the relative energies. There is an additional Doppler shift associated with the transformation between reference frames, which gets canceled when we look at the energy differences. Therefore, for the purpose of our experiments, momentum and detuning are equivalent up to a numerical factor: $\Delta_0/E_R=4q_x/k_R$, and we can measure by preparing a zero momentum state and measuring the probability amplitude for different values of Raman detuning $\Delta_0$. 

%In order to maximize our signal to noise ratio (SNR) and minimize the required number of data points we use some fancy algorithm that I'm still not sure which one will work best. We also choose the spacing and the total number of pulses for each spectra so that the bandwith and resolution of the Fourier transform allow us to resolve the frequencies of interest. 

\subsection{Spectroscopy of a driven system}

We can use the same principles described in the previous section to study more complex time dependent systems. A particularly interesting case is that of periodically driven systems [cita aqui], which can be described by effective coupling terms in the Hamiltonian that arise from averaging the fast dynamics of the system. The Fourier spectroscopy technique can be applied to such systems to measure the Floquet quasi-energy spectrum and also study the couplings within different Floquet manifolds. 

We will focus on the case of a spin-1 spin-orbit coupled system that is coupled by a multiple frequency Raman field, as shown in Fig 3b. The interference of the multiple frequencies leads to a periodic amplitude modulation in the field and an effective Floquet Hamiltonian that has tunable spin-orbit coupling. 

We add two sidebands to one Raman beam at angular frequencies $\omega=\omega_L+\omega_Z+\Delta_0 \pm \delta\omega$.  The Hamiltonian in Eq.\ref{Eq:SOCone} remains unchanged, except for the coupling strength that takes the form $	\Omega_R(t)=\Omega_0 + \Omega\cos(\delta\omega t)$. Our periodically driven system is well described by Floquet theory: the eigenstates are of the form $\ket{\Psi(t)}=\sum_{j}c_je^{i\epsilon_jt}\ket{u_j(t)}$ where $\ket{u(t)}$ are time-periodic states and the $\epsilon_j$ are the Floquet quasi-energies, wich are  $\epsilon_j,n=\epsilon_{j,m} + (n-m)2\pi/T$

\begin{figure*}
	\begin{center}
		\includegraphics{figures/fig3v1.pdf}
		\caption 
		{
			Time evolution of the BEC for Raman pulsing times between 5 and 10 $\mu s$, for different spin orbit coupling regimes:
			{\bf (i)} $\Omega_0=9.9 E_L$, $\Omega=04$,  $\Delta=5.8 E_L$, 
			{\bf (ii)} $\Omega_0=0$, $\Omega=8.6 E_L$,  $\Delta=-0.7 E_L$, and
			{\bf (iii)} $\Omega_0=1.5 E_L$, $\Omega=8.4 E_L$,  $\Delta=-4.7 E_L$
			
		}
		\label{fig:Figure3}
	\end{center}
\end{figure*}

The time evolution of the system after one driving cycle can be described in terms of an effective time-independent Hamiltonian $e^{iT\hat{H}_{eff}}$. If $\delta\omega \gg \epsilon$ and $\delta\omega \gg 4E_L$, this effective Floquet Hamiltonian retains the form of \ref{Eq:SOCone} with renormalized coefficients, and an additional term that explicitly couples the $m_f=-1$ and $m_f=+1$ states:

\begin{align}
	\begin{split}
		\hat{H} = &\frac{\hbar^2\hat{k}^2}{2m} + \alpha\hat{k}\hat{F}_z +4E_L\mathbb{I} + \frac{\Omega_0}{2}\hat{F}_x \\
		&+ \frac{\tilde{\Omega}}{2}\hat{F}_{xz} +(\tilde{\epsilon}+4E_L)(\hat{F}_z^2-\mathbb{I}) +\tilde{\Delta}\hat{F}_z, 
		\label{Eq:SOCeff}
	\end{split}
\end{align}	
%
with $\alpha= J_0(\Omega/2\delta\omega)\alpha_0$, $\tilde{\Omega}=1/4(\epsilon+4E_L) (J_0(\Omega/\delta\omega)-1)$, $\Delta=J_0(\Omega/2\delta\omega)\Delta_0$, and $\tilde{\epsilon}= 1/4(4E_L-\epsilon) - 
1/4(4E_L + 3 \epsilon) J_0( \Omega/\delta\omega)$


There are two limiting cases of this effective Hamiltonian \ref{Eq:SOCeff} which will be of interest: (1) for large quadratic Zeeman shift the system can be described as an effective spin $1/2$ (cite Lindsay) system where the spin orbit coupling strength and the Raman coupling can be independently tuned and 



%In the high field regime, when $\epsilon > 4E_R$, the $m_f=-1\leftrightarrow m_f=0$ and $m_f=0 \leftrightarrow m_f=+1$ transition cannot be resonantly addressed with the same frequency. By adiabatically eliminating the $m_f=0$ state we can describe the system in terms of an effective spin $1/2$  with an effective Hamiltonian

\begin{align}
	\begin{split}
		\hat{H}_{eff} = & \frac{\hbar^2}{2m}(\hat{k}+2\kr\hat{\sigma}_z)^2 + \frac{\hbar\Omega'}{2}\hat{\sigma}_x  +\Delta\hat{\sigma}_z  
		\label{Eq:SOChalf}
	\end{split}
\end{align}	

where we have defined an effective coupling between the $m_f=-1$ and $m_f=+1$ states $\Omega'=\tilde{\Omega}+\hbar\Omega_0^2/2(\tilde{\epsilon})$. 


Fig1b shows the high field dispersion relation, both for the modulated and unmodulated cases (here goes an image of bands). The minima, originally locted at $\pm2 k_L$, are shifted and the size of the spin-orbit gap is changed for different choices of $\Omega_0$, $\Omega$, and $\delta\omega$.  










\section{Methods}

\subsection{Modulated/tripple frequency coupling (theory)}


\subsection{Fourier Spectroscopy}		

The Fourier Spectroscopy technique takes advantage of the time evolution of a bare atomic state after a dressing field is suddenly turned. The initial state becomes a superposition of dressed states and it undergoes Rabi oscillations in time. The spectral components of these oscillations contain information about the energies of the dressed states. 

Aquí algo sobre el caso particular de las energias SOC.

The measurement can be simplified by noticing that a non-moving atom cloud in the laboratory reference frame dressed by a field with non-zero detuning is equivalent to a moving cloud with a resonant field in a suitable moving reference frame. As can also be seen in the Hamiltonian (citarlo aqui)  the detuning term $\delta/Er$  and the momentum term $4 k/k_R$ have the same effect in the energy differences.


For the case of our spin-orbit coupled BECs, the bare state
The system is let to evolve for a finite time T and afterwards the field is snapped off. A Stern-Gerlach pulse applied at our 21 ms time of flight (TOF) allows us to project the state of the condensate back into the bare $m_f$ basis. 





For the experimental sequence we start 




\section{Results}
Put a plot with time evolution and fit. One for simple soc case, one for Molmer-Sorensen case. Fourier trasform and spectra. One image that shows how it is done and then several other spectra. It would be great if I show both tunability and and Floquet bands. Effective mass measurement data? Show higher harmonics. Spin resolved spectroscopy. 



\section{Discussion}


Heating due to scattering of spontaneously emitted photons is always present in our system. It is also well known that heating is present in periodically driven systems, and while it can be minimized by increasing the driving frequency, it in exchange requires more Raman power to tune the spin-orbit coupling strength. The time scales of our pulsing experiments never exceeded $900\us$ which is small compared to the lifetime of our system

Something about it not being necessary to fit to a complicated model 

In conclusion, we can measure the spin and momentum dependent dispersion relation for a spin-1 spin-orbit coupled BEC using our Fourier spectroscopy technique. When used to study periodically driven systems, we are able tp see a rich spectrum that arises from the Floquet quasi-energies. This method is good for any effective three level system with a quadratic branch in the dispersion relation and does not require any additional hardware as it relies only on the Hamiltonian evolution of the system.  This technique might prove particularly useful to probe the spin-resolved energy dispersion of atoms in the presence of Rashba spin-orbit coupling using newly proposed schemes to generate this type of coupling without the use of excited states and could lead to a better understanding of new topological matter. 


\section{Conclusion}



\section{Appendix A}


\subsection{Effective SOC Hamiltonian}	
%Probably too much. 
The effective spin-orbit coupling Hamiltonian can be derived from the electric dipole Hamiltonian describing the interaction between our Rb atoms and the multiple frequency Raman lasers


\begin{align}
	\hat{H}_{AL}(t) = \left[\Omega_{21}\cos(2\kr x-\omega_{21} t +\phi_1)+\Omega_{31}\cos(2\kr x-\omega_{31} t+\phi_2)+\Omega_{41}\cos(2\kr x-\omega_{41} t + \phi_3)\right]\hat{F_x}
\end{align}
%
where $\Omega_{ij}\propto \vec{E_i}\times\vec{E_j^{\star}}$ represents the coupling strength associated to each pair of Raman beams and $\omega_{ij} = \omega_{i}-\omega_{j} $. We choose the frequencies so that $\omega_{31} + \omega{21}$ is at 4 photon resonance with the $m_f = +1\rightarrow m_f = -1$ transition. We then apply a rotation about the z axis $\hat{U} = e^{i\bar{\omega} t\hat{F}_z}$ where $\bar{\omega} = \frac{\omega_{21}+\omega_{31}}{2}$. For the choice of parameters 
$\Omega_{21} =\Omega_{31} = \Omega $, $\Omega_{41}=\Omega_0$, and $\omega_{41} =\frac{\omega_{21}+\omega_{31}}{2}$,and after applying the rotating wave approximation (RWA), the Hamiltonian transforms to  
%
%

	
\begin{align}
	\begin{split}
		\hat{\tilde{H}}_{AL}(t) &= \frac{1}{2}\lbrace\Omega\cos(2\kr x+\delta\omega t +\phi_1)+\Omega\cos(2\kr x-\delta\omega t + \phi_2) +\Omega_0\cos(2\kr x + \phi_3)\rbrace \hat{F}_x \\
		& -  \frac{1}{2}\lbrace\Omega\sin(2\kr x+\delta\omega t +\phi_1)+\Omega\sin(2\kr x-\delta\omega t +\phi_2)+\Omega_0\sin(2\kr x +\phi_3)\rbrace \hat{F}_y
	\end{split}
\end{align}

we have the freedom of defining the time origin so we can get rid of one phase, lets make it $\phi_3$ for convenience. Can I get rid of a second phase? Make $\phi_1 = -\phi_2$
 
 %
 %
\begin{align}
 	\begin{split}
 		\hat{\tilde{H}}_{AL}(t)  = [\frac{\Omega_0}{2} + \Omega\cos(\delta\omega t+\phi_1)][\cos(2\kr x)\hat{F}_x - \sin(2\kr x)\hat{F}_y],
 	\end{split}
\end{align}


where we have defined $\delta\omega=\frac{\omega_{31}-\omega_{21}}{2}$. This Hamiltonian term describes a helically precessing effective Zeeman field with amplitude oscillating periodically in time. 

The complete Hamiltonian can therefore be written as

\begin{align}
	\begin{split}
		\hat{H}(t)=\frac{\hbar^2}{2m}\hat{k}^2 + \mu\mathbf{B}_{eff}\cdot\hat{\mathbf{F}}
		\label{Eq:Beff}
	\end{split}
\end{align}

(check units here)
with $\mu\mathbf{B}_{eff}=(\frac{\Omega_0}{2} + \Omega\cos(\delta\omega t+\phi_1))(\cos(2\kr x)\mathbf{e}_x-\sin(2\kr x)\mathbf{e}_y)+\Delta_0\mathbf{e}_z$


We can apply a position dependent rotation $\hat{U} = e^{i2\kr x \hat{F}_z}$ transforms our Hamiltonian into the form of Eq. \ref{Eq:SOCone} with a time dependent Raman coupling.

\begin{align}
	\begin{split}
		\hat{H}(t) = & \frac{\hbar^2}{2m}(\hat{k}-2\kr\hat{F}_z)^2 + (\frac{\Omega_0}{2} + \Omega\cos(\delta\omega t))\hat{F}_x \\& + \epsilon(\hat{F}_z^2-\mathbb{I})+\Delta\hat{F}_z \\
		=&\frac{\hbar^2\hat{k}^2}{2m} + \alpha_0\hat{k}\hat{F}_z +4E_L\mathbb{I} + \frac{\Omega(t)}{2}\hat{F}_x\\
		& +(\epsilon+4E_L)(\hat{F}_z^2-\mathbb{I}) +\Delta_0\hat{F}_z 
	\end{split}
\end{align}

(check factors of 2!)

To get rid of the time dependence in the Hamiltonian and ultimately getting the `tunable' spin-orbit coupling we can choose a transoformation of the Hamiltonian such that $\hat{U}^{\dagger} \frac{\partial\hat{U}}{\partial t} = -i \frac{\Omega(t)}{2}\hat{F}_x$. This will be satisfied for

\begin{align}
	\hat{U} = e^{-i\frac{\Omega}{2}\int_0^t\cos(\delta\omega t')dt'} = e^{-i\frac{\Omega}{2\delta\omega}\sin(\delta\omega t)}.
\end{align}

Under this time dependent transformation, the time evolution of the system will be given by the Hamiltonian
%

\begin{align}
	\begin{split}
		\hat{\tilde{H}} = & \hat{U}^{\dagger}\hat{H}(t)\hat{U} + i\hat{U}^{\dagger} \frac{\partial\hat{U}}{\partial t} \\
		 = &\frac{\hbar^2\hat{k}^2}{2m} + \alpha\hat{k}\hat{F}_z +4E_L\mathbb{I} + \frac{\Omega_0}{2}\hat{F}_x \\
		&+ \frac{\tilde{\Omega}}{2}\hat{F}_{xz} +(\tilde{\epsilon}+4E_L)(\hat{F}_z^2-\mathbb{I}) +\Delta\hat{F}_z 	
	\end{split}
\end{align}
%
%
which is exactly Eq. \ref{Eq:SOCeff}. To arrive to this final form we have transformed the operators that don't commute with $\hat{F}_x$ as

%
\begin{align}
	\begin{split}
		e^{i\theta \hat{F}_x} \hat{F}_z e^{-i\theta \hat{F}_x}=& \cos\theta \hat{F}_z + \sin\theta\hat{F}_y \\
		e^{i\theta \hat{F}_x} \hat{F}_y e^{-i\theta \hat{F}_x} =& -\sin\theta\hat{F}_z +\cos\theta\hat{F}_y \\
		e^{i\theta \hat{F}_x} \hat{F}_z^2 e^{-i\theta \hat{F}_x} = &\cos^2\theta\hat{F}_z^2+\sin^2\theta\hat{F}_y^2 + \sin\theta\cos\theta(\hat{F}_z\hat{F}_y + \hat{F}_y\hat{F}_z).
	\end{split}
\end{align}
%
and neglected the terms oscillating at high frequency

\begin{align}
	\begin{split}
	\cos(\Omega/2\delta\omega\sin(\delta\omega t))&= J_0(\Omega/2\delta\omega) + 2\sum_{n=1}^{\infty}J_{2n}(\Omega/2\delta\omega)\cos(2n(\delta\omega t)) \\
	& \approx J_0(\Omega/2\delta\omega) \\
	\sin(\Omega/2\delta\omega\sin(\delta\omega t))&= 2\sum_{n=0}^{\infty}J_{2n+1}(\Omega/2\delta\omega)\sin((2n+1)(\delta\omega t)) \approx 0,
	\end{split}
\end{align} 


(verify how large $\delta\omega$ needs to be for this approximation to be valid)
%
%
 
\subsection{spin half system}
\begin{align}
\begin{split}
\hat{H}_{eff} = & \frac{\hbar^2}{2m}(\hat{k}+2\kr\hat{\sigma}_z)^2 + \frac{\hbar\Omega'}{2}\hat{\sigma}_x  +\Delta\hat{\sigma}_z  
\label{Eq:SOChalf}
\end{split}
\end{align}	

where we have defined an effective coupling between the $m_f=-1$ and $m_f=+1$ states $\Omega'=\tilde{\Omega}+\hbar\Omega_0^2/2(\tilde{\epsilon})$. 



\subsection{Relative phases}

Here put the Hamiltonian with phases.


The relation between the dynamics/time evolution of a system is rooted in the heart of quantum mechanics. 






Does driving strength exceed transition frequency? No, but it does exceed the quadratic zeeman shift. 
From the floquet paper: The observed  system dynamics is very well described in terms of
quasienergies and quasienergy states, as predicted by
Floquet theory. In particular, we observe several frequency
components in the dynamics, in very good agreement with
theory
Think about the coupling strength vs quadratic zeeman shift in terms of excited floquet states. 
Think about fourier spectroscopy for Bloch bands.
In analogy to solid state stuff we introduce a new Fourier based spectroscopy technique that we use to measure the spin dependent energy-momentum dispersion bands of the system. With the addition of a one dimensional optical lattice, this work opens the ground for measurements of the Hofstadter butterfly spectrum. 


Previous studies have shown that strong modulation in the Raman coupling strength for an effective spin $1/2$ system leads to tunable spin-orbit coupling strength. In addition to this, the use of two Raman coupling frequencies in a spin one system which is equivalent to a single frequency amplitude modulated coupling leads to new magnetic phases (cite spin one papar). Here we extend the study of the effects of amplitude modulation of the Raman coupling strength/multiple frequency couplings. We will show that for a spin one system, we can independently  tune the spin orbit-coupling gap and strength, and we can additionally engineer a cyclic coupling between the three $m_F$ magnetic sub-levels. 



Fixes:
\begin{itemize}
	\item Write frequencies as $\omega_L$ and $\omega_L +\Delta\omega \pm\delta\omega$. Is there too many $\delta$ symbols, confusing?
	\item Call $J_0$ the zeroth order Bessel functions of the first kind.
	\item don't say modulation or multiple frequencies, just say we amplitude modulate by using 
	\item no effective model, say the floquet Hamiltonian takes de form
	\item mention 2 paths
\end{itemize}


effective Hamiltonian that captures the essential characteristics
of the modulated system. This strategy exploits the
fact that modulation schemes can be tailored in such a way
that effective Hamiltonians reproduce the Hamiltonians
of interesting static systems.

Corrections of the effective Hamiltonian are of the order $1/\delta\omega$.

%% Here is the endmatter stuff: Supplementary Info, etc.
%% Use \item's to separate, default label is "Acknowledgments"

%\begin{addendum}
%\item We appreciate useful discussions with Stefan Natu.  This work was partially supported by the ARO's atomtronics MURI, by the AFOSR's Quantum Matter MURI, NIST, and the NSF through the PFC at the JQI.
 
%\item[Author Contributions] All authors excepting I.B.S contributed to the data taking effort.  All authors: analyzed data; performed numerical and analytical calculations; and contributed to writing the manuscript.  I.B.S. proposed the initial experiment (with great enthusiasm, but with little appreciation for the many technical hurdles to be resolved by the remainder of the team).

%\item[Correspondence] Correspondence and requests for materials should be addressed to I.B.S.~ (email: ian.spielman@nist.gov).

%\bibliography{MolmerSorensen}

\end{document}
