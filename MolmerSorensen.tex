\documentclass[12pt]{iopart}
%[amsmath,amssymb,floatfix]

\usepackage{iopams}
\expandafter\let\csname equation*\endcsname\relax
\expandafter\let\csname endequation*\endcsname\relax
\usepackage{amsmath}


\usepackage{graphicx}       % Include figure files
\newcommand{\documentpath}{./Figures}
\include{macros}

\begin{document}
		
\title{Fourier spectroscopy of a spin-orbit coupled Bose gas}
	
\author{Ana Vald\'es-Curiel, Dimitri Trypogeorgos, Erin E. Marshall, Ian B. Spielman}
\address{Joint Quantum Institute, University of Maryland and National Institute of Standards and Technology, College Park, Maryland, 20742, USA}
\date{\today}

\begin{abstract}
	%Here we look at soc-ed bose gas with modulation. Technique generalizes the proposal of (Karina) and opens the door to generating cyclic couplings between $m_f=+1$ and $m_f=+1$ we propose a new method of probing the energy-momentum dispersion of a spin-orbit coupled quantum gas
	
%	The coupling between an electron's spin and motion lies withing the heart cool things in condensed matter systems. The high degree of control in cold atoms systems makes them ideal candidates for studying new phases of matter and simulating quantum systems.
	 We propose a time domain technique to measure the band structure of a spin-1 spin-orbit coupled Bose-Einstein condensate that relies on the Hamiltonian evolution of the system. We drive transitions at different values of detuning from Raman resonance and extract the Fourier components of the time dependent evolution to reconstruct the spin and momentum dependent energy spectrum. We add a periodic modulation to one Raman field which results in a tunable spin-orbit coupling dispersion and a spectrum of Floquet quasi-energies that we can directly measure, showing the robustness of our technique.  
\end{abstract}

\maketitle
%\tableofcontents

\section{Introduction}
%Empezar con porque nos interesa soc (tal tez hacer enfasis en edge states, topological stuff, hofstadter). Describir el Hamiltoniano para un sistema de spin uno. Añadir la modulación  y explicar que modulación es equivalente a tres frequencias (citar spin one paper) espectralmente como en ciertos límites puedes tener acoplamiento cíclico. Por que quiero medir directamente las bandas? Mencionar que hay otras tecnicas como spin injection, pero que esta tecnica ofrece la ventaja de que no necesitas hardware adicional o "reservoir states" que solo ciertos elementos tienen. Motivar eso. De ahí voy a Fourier spectroscopy, funciona para un sistema conocido. Hamiltoniano modulato, comparar el gap de Molmer-Sorensen (acoplamiento entre -1 y +1) con la transición de dos fotones. Pensar en Mariposas de Hofstater, fin. 


%Point to sell: modulations are cool and deep in the heart of atomic physics. Maybe talk about pulsed optical lattices? Modulated Raman stuff from the group with cool frequency ramps? Read Dalibards paper on modulated systems for inspiration, Read ians paper on detecting topological matter using cold atoms for more inspiration.


Spin–orbit (SO) coupling is an essential mechanism for most
spintronics devices and leads to many fundamental phenomena
in condensed matter physics and atomic physics. For example,
SO coupling gives rise to the quantum spin Hall effect in
electronic condensed matter systems

Start with corny paragraph, about soc and how cold atoms are awesome.
Describe SOC at high field and at four photon resonance. Describe Fourier spectroscopy. Describe modulated Raman. 



\subsection{Fourier spectroscopy of spin-orbit coupled atoms}		


%Think transfer functions and spectroscopy. Spectroscopy is a vertical cut, it looks at response of the system when driven with frequencies within the Fourier limited bandwith of the pulse. 


%We consider a spin one system in a uniform magnetic field  $B\mathbf{e}_z$ that Zeeman splits the energy levels by $\omega_Z/2\pi=g_f\mu_BB = 14 MHz$. A quadratic Zeeman shift $\epsilon$ breaks the $m_F=\pm1\leftrightarrow m_F=0$ degeneracy. A pair of intersecting Raman lasers with frequency relation $\omega_A-\omega_B=\omega_Z + \Delta$, where $\omega_Z$ is the linear Zeeman shift and $\Delta$ is the detuning from four photon resonance. 


\begin{align}
\begin{split}
\hat{H} = &\frac{\hbar^2\hat{k}^2}{2m} + \alpha_0\hat{k}\hat{F}_z +4E_L\mathbb{I} + \frac{\Omega_R}{2}\hat{F}_x\\
& +(\epsilon+4E_L)(\hat{F}_z^2-\mathbb{I}) +\Delta_0\hat{F}_z, 
\label{Eq:SOCone}
\end{split}]
\end{align}	


where we have introduced the natural units of our system: the single photon recoil momentum $k_L=\frac{2\pi}{\lambda_R}\sin(\theta/2)$ and its associated recoil energy $E_L=\frac{\hbar^2k_L^2}{2m}$, determined by the wavelength and geometry of the Raman field. We have additionally introduced the spin-orbit coupling strength $\alpha_0=\frac{\hbar^2k_L}{m}$ and the Raman coupling strength $\Omega_R\propto E_A^{\star}E_B$ which is proportional to the field intensity. 


In order to explicitly measure the  energy-momentum dispersion relation, we will use a Fourier based spectroscopy technique, which relies on the time evolution of an atomic state after a dressing field is suddenly turned on, and the initially bare states become superpositions of dressed states undergoing Rabi oscillations in time  with spectral components related to the relative energies of the dressed states. 

%as well as for the future measurement of the Hofstadter Butterfly spectrum%


For the case of a spin-orbit coupled atomic system, the dressed state energies are explicitly dependent on both the particle's spin and momentum. Therefore, in order to fully characterize the energy-momentum dispersion we must prepare an atomic state at a given spin and momentum $\ket{k_i,m_{f_i}}$, pulse on the Raman field, and measure the time evolution.  

In practice it is not as straightforward to reliably prepare an arbitrary momentum state in the lab. The measurement however can be simplified by noticing that a non-moving atom cloud in the laboratory reference frame dressed by a field with non-zero detuning is equivalent to a moving cloud with a resonant field in a suitable moving reference frame. This can be explicitly seen in the Hamiltonian \ref{Eq:SOCone} where the detuning term $\delta/Er$  and the momentum term $4 k/k_R$ have the same effect in the relative energies. There is an additional Doppler shift associated with the transformation between reference frames, which gets canceled when we look at the energy differences. Therefore, for the purpose of our experiments, momentum and detuning are equivalent up to a numerical factor. 

The method described above only allows us to measure relative energies and we must add a known energy reference if we want to recover the dispersion relation. We can do so by measuring the effective mass $m^{\star} = \hbar^2[\frac{d^2E(k_x)}{dk_x}]^{-1}$ of the nearly quadratic lowest branch of the dispersion, and then shifting the measured frequencies accordingly. 

Here an image of dispersion, time evolution, FFT, spectrum of energy differences and reconstructed spectrum.

In order to maximize our signal to noise ratio (SNR) and minimize the required number of data points we use some fancy algorithm that I'm still not sure which one will work best. We also choose the spacing and the total number of pulses for each spectra so that the bandwith and resolution of the Fourier transform allow us to resolve the frequencies of interest. 



\subsection{Spectroscopy of a driven system}


Previous studies have shown that periodically driven systems such as cold atoms in driven optical fields[karina, optical lattices, germany group] exhibit effective coupling terms in the Hamiltonian that arise from averaging the fast dynamics of the system. Our Fourier based spectroscopy can also be used to study such systems as it can reveal additional information about the system's Floquet quasi-energy spectrum. We will focus on the case of a spin-1 spin-orbit coupled system that is coupled by a multiple frequency Raman field, as shown in Fig 3b. The interference of the multiple frequencies lead to a periodic amplitude modulation in the field and an effective Floquet Hamiltonian that has tunable spin-orbit coupling. 

If we add two sidebands to one Raman beam, offset from the carrier frequency by $\pm \delta\omega$ and, for simplicity, we choose al the relative phases to be zero, then the Hamiltonian in Eq.\ref{Eq:SOCone} remains unchanged, except for the coupling strength that takes the form $	\Omega_R(t)=\Omega_0 + \Omega\cos(\delta\omega t)$. Our periodically driven system is well described by Floquet theory: $\ket{\Psi(t)}=\sum_{j}c_je^{i\epsilon_jt}\ket{u_j(t)}$ where $\ket{u(t)}$ are time-periodic Floquet states and the $\epsilon_j$ are the Floquet quasi-energies, wich are  $\epsilon_j,n=\epsilon_{j,m} + (n-m)2\pi/T$

The time evolution of the system after one driving cycle cab described in terms of an effective time-independent Floquet Hamiltonian $e^{iT\hat{H}_{eff}}$. If $\delta\omega \gg \epsilon$ and $\delta\omega \gg 4E_L$, this effective Floquet Hamiltonian retains the form of \ref{Eq:SOCone} with renormalized coefficients, and an additional term that explicitly couples the $m_f=-1$ and $m_f=+1$ states:

\begin{align}
	\begin{split}
		\hat{H} = &\frac{\hbar^2\hat{k}^2}{2m} + \alpha\hat{k}\hat{F}_z +4E_L\mathbb{I} + \frac{\Omega_0}{2}\hat{F}_x \\
		&+ \frac{\tilde{\Omega}}{2}\hat{F}_{xz} +(\tilde{\epsilon}+4E_L)(\hat{F}_z^2-\mathbb{I}) +\tilde{\Delta}\hat{F}_z, 
		\label{Eq:SOCeff}
	\end{split}
\end{align}	
%
with $\alpha= J_0(\Omega/2\delta\omega)\alpha_0$, $\tilde{\Omega}=1/4(\epsilon+4E_L) (J_0(\Omega/\delta\omega)-1)$, $\Delta=J_0(\Omega/2\delta\omega)\Delta_0$, and $\tilde{\epsilon}= 1/4(4E_L-\epsilon) - 
1/4(4E_L + 3 \epsilon) J_0( \Omega/\delta\omega)$


There are two limiting cases of this effective Hamiltonian \ref{Eq:SOCeff} which will be of interest: (1) for large quadratic Zeeman shift the system can be described as an effective spin $1/2$ (cite Lindsay) system where the spin orbit coupling strength and the Raman coupling can be independently tuned and 



%In the high field regime, when $\epsilon > 4E_R$, the $m_f=-1\leftrightarrow m_f=0$ and $m_f=0 \leftrightarrow m_f=+1$ transition cannot be resonantly addressed with the same frequency. By adiabatically eliminating the $m_f=0$ state we can describe the system in terms of an effective spin $1/2$  with an effective Hamiltonian

\begin{align}
	\begin{split}
		\hat{H}_{eff} = & \frac{\hbar^2}{2m}(\hat{k}+2\kr\hat{\sigma}_z)^2 + \frac{\hbar\Omega'}{2}\hat{\sigma}_x  +\Delta\hat{\sigma}_z  
		\label{Eq:SOChalf}
	\end{split}
\end{align}	

where we have defined an effective coupling between the $m_f=-1$ and $m_f=+1$ states $\Omega'=\tilde{\Omega}+\hbar\Omega_0^2/2(\tilde{\epsilon})$. 


Fig1b shows the high field dispersion relation, both for the modulated and unmodulated cases (here goes an image of bands). The minima, originally locted at $\pm2 k_L$, are shifted and the size of the spin-orbit gap is changed for different choices of $\Omega_0$, $\Omega$, and $\delta\omega$.  










\section{Experiment}

We are interested in measuring the SOC dispersion for three different types of coupling schemes: (i) A time independent spin-orbit coupled system, $\Omega_0\neq0$ and $\Omega=0$, (ii) for a periodically driven spin-orbit coupled system with no DC offset $\Omega\neq0$ and $\Omega_0=0$, and (iii) A time periodically driven spin-orbit coupled system with a DC offset $\Omega\neq0$ and $\Omega_0\neq0$. 

\subsection{Spectroscopy experimental sequence}


We start our experiments with a Rb$^{87}$ Bose-Einstein condensate \cite{lin_rapid_2009} (BEC) with $N\approx 4\times 10^4$ (measure) atoms in the $\ket{F=1,m_F=0}$ state, confined in a $1064\nm$ crossed optical dipole trap, with trapping frequencies $(\omega_x,\omega_y,\omega_z)=2\pi(42(3),34(2),133(3))\ \Hz$. We break the degeneracy between the $m_F$ magnetic sub-levels by applying a $17.0556$ G bias field along the z axis, which produces a Zeeman splitting of $12 \ \MHz$ and a quadratic Zeeman shift that lowers the energy of the $\ket{F=1,m_F=0}$ state  by $20.9851$ kHz. We adiabatically prepare our BEC in the $\ket{m_F=0}$ by slowly ramping the bias field while applying a $12\MHz$ radio-frequency field. We then apply a pair of microwave pulses that serve to monitor and stabilize the bias field and. We generate spin-orbit coupling between the magnetic sub levels with a pair of intersecting, cross-polarized Raman beams, with wavelength $\lambda=790.024 nm$ propagating along $\mathbf{e}_x+\mathbf{y}$ and $\mathbf{e}_x-\mathbf{e}_y$ as shown in Fig 1a. We offset the frequency of the beams using two acusto optic modulators (AOMs), one of them driven by a superposition of up to three different frequencies. On resonance, the laser frequencies satisfy the condition $\omega_A-\omega_B=\omega_A-\frac{\omega_{B+}+\omega_{B-}}{2}=\omega_Z$, and we change the Raman detuning $\Delta_0$ by keeping the magnetic field constant and changing the value of the frequency $\omega_A$.

To get our probability amplitude measurements, we keep the detuning value $\Delta_0$ fixed and pulse the Raman beams on for time intervals between $5\us$ and up to $900\us$. For the time independent SOC measurements (case (i)) we take a total of 120 different pulses, for the periodically driven SOC measurements (cases (ii) and (iii)) which require better resolution and bandwidth, we take a total of 180 pulses. After pulsing the Raman we release our atoms from the optical dipole trap and let them fall for a $21$ ms time of flight (TOF) time before and apply a spin-dependent force using magnetic field gradient. Our absorption images reveal the atoms spin and momentum distribution, from which we can extract the probability amplitudes by counting the fractional number of atoms in each spin and quasimomentum state. We repeat this procedure for values of Raman detuning within the interval $\pm 12 E_L$ which corresponds to quasimomentum values within $\pm 3k_L$.

The time dependent SOC measurements additionally required phase stability between the 3 frequency components in the Raman B field. We set all the relative phases to zero at the beginning of each pulse and kept it constant throughout our experiments. We made the choice of zero relative phase as it maximizes the effective couplings $\Omega$ and $\Omega_0$ for a given field intensity. For a more detailed discussion of the effect of the relative phases see the Appendix section (?). 

\subsection{Effective mass measurement}

We measure the atom's ground state effective mass by adiabatically preparing our BECs in the lowest eigenstate and inducing dipole oscillations.The effective mass $m^{\star}$ of the dressed atoms  is related to the bare mass $m$ and the bare and dressed trapping frequencies $\omega$ and $\omega^{\star}$ by the ratio $m^{\star}/m=\sqrt{\omega^{\star}/\omega}$. We prepare our system in the  $\ket{m_F=0,\ k_x=0}$ and adiabatically turn on the Raman in $\approx10$ ms while also ramping the detuning to a non-zero value, around $0.5\Er$. Our system does not have the capability to dynamically change the laser frequency while maintaining phase stability, so unlike the pulsing experiments, we ramped the magnetic field to change the resonance conditions. This detuning shifts the minima in the ground state energy away from zero quasi-momentum. We then suddenly snap the field back to resonance which changes the equilibrium conditions of the system and excites the dipole mode of our optical dipole trap. To measure the bare state frequency, we use the Raman beams to initially excite the dipole mode of the trap but we quickly turn off the field ($\sim1\ms$) and let the BEC continue to oscillate in the unmodified dipole potential. 

For this set of measurements we modified our trapping frequencies to $(\omega_x, \omega_y, \omega_z)=2\pi(35.9, 32.5, xx) \Hz$  so that they were nominally symmetric along the $x-y$ plane. 

\subsection{Magnetic field stabilization}
We stabilized the magnetic field and measured fluctuations about the desired set point by applying a pair of microwave pulses with frequencies close to resonance from the $5^2{\rm S}_{1/2}$ $F=2$ state, and imaging the in-situ the population transfered by each pulse. 

We first prepare our BEC in the $\ket{F=1, m_f=0}$ state and apply a $17.0556$ G bias filed along the z axis. We then apply a pair of $250\mu s$ microwave pulses close to $6.83 GHz$ that transfers about $10\%$ of the atoms into the $F=2$ manifold. The pulses were detuned by $\pm 2\kHz$ from the $\ket{F=1,m_F=0}\leftrightarrow\ket{F=2,m_F=1}$ transition and were spaced in time by 2 periods of 60 Hz. We image the atoms transfered into $F=2$ non-destructively using absorption imaging without repumping light. The imbalance in the number of atoms transferred by each pulse gives us a $4\kHz$ wide error signal that we use both to feed forward our bias coils for active field stabilization, and also to keep track of the magnetic fields at each shot. We trigger our sequence to the line and both the microwave and Raman pulses are timed at integer periods of $60\Hz$ and performed at the zero-derivative point of the $60\Hz$ curve in order to minimize additional magnetic field fluctuations

%Just before transferring the BEC into |mF = −1, kx = 2kRi, two 6.8 GHz microwave pulses
%spaced in time by 50 ms each out-coupled ≈10% of the atoms to the f = 2 hyperfine manifold.
%These atoms were separately imaged (without repumping on the f = 1–2 transition) leaving
%the atoms in f = 1 undisturbed. These f = 2 atoms served two purposes: (i) by setting the
%microwave frequency 2 kHz above (first pulse) and 2 kHz below (second pulse) resonance, we
%tracked shifts in the bias field that would change our four-photon Raman resonance condition.
%Upon analysing the data, we rejected points where the atom number difference between these
%two images was greater than two standard deviations from being equal; (ii) we determined
%the BEC’s position immediately before each zitterbewegung experiment began, allowing us
%to cancel shot-to-shot variations in the trap position. The beginning of the three transfer
%pulses—two microwave outcoupling pulses, and the final four-photon Raman pulse—were each
%separated in time by 50 ms. As three periods of a 60 Hz cycle, this separation was chosen to
%reduce magnetic field background fluctuations at the power line frequency, and to facilitate
%rethermalization between pulses.


\section{Results}

We studied the time evolution for three different schemes of SOC. 
We can fit our experimentally calculated probability amplitudes to the Hamiltonian evolution. This can serve two purposes

We calibrated the Raman coupling terms $\Omega$ and $\Omega_0$ and the detuning from Raman resonance $\Delta$ by fitting the three-level Rabi oscillations of the $\ket{m_F=0}$ and $\ket{m_f=\pm 1}$ states to the time evolution given by the Hamiltonian in Eq. \ref{Eq:SOCone}. Figure \ref{fig:Figure3} shows representative traces for the time evolution of our system for the three cases outlined above. These calibrations along with the information gained from the images of atoms out-coupled  with microwaves guaranteed that we were indeed at the correct detuning from Raman resonance and that the Raman coupling strength remained nominally constant throughout our measurements. 

\begin{figure*}
	\begin{center}
		\includegraphics{figures/tdse.pdf}
		\caption 
		{
		Time evolution of the BEC for Raman pulsing times between 5 and 10 $\mu s$, for different spin orbit coupling regimes:
		{\bf (i)} $\Omega_0=9.9 E_L$, $\Omega=04$,  $\Delta=5.8 E_L$, 
		{\bf (ii)} $\Omega_0=0$, $\Omega=8.6 E_L$,  $\Delta=-0.7 E_L$, and
		{\bf (iii)} $\Omega_0=1.5 E_L$, $\Omega=8.4 E_L$,  $\Delta=-4.7 E_L$
			
		}
		\label{fig:Figure3}
	\end{center}
\end{figure*}



The time evolution shows higher frequency components for cases (ii) and (iii), as expected from the Floquet quasi-energy spectrum, since the Raman coupling strength $\Omega, \Omega_R$ is comparable to the separation between the Floquet manifolds $\epsilon$. 


\begin{figure*}
	\begin{center}
		\includegraphics{figures/bands.pdf}
		\caption 
		{
			Time evolution of the BEC for Raman pulsing times between 5 and 10 $\mu s$, for different spin orbit coupling regimes:
			{\bf (i)} $\Omega_0=9.9 E_L$, $\Omega=04$,  $\Delta=5.8 E_L$, 
			{\bf (ii)} $\Omega_0=0$, $\Omega=8.6 E_L$,  $\Delta=-0.7 E_L$, and
			{\bf (iii)} $\Omega_0=1.5 E_L$, $\Omega=8.4 E_L$,  $\Delta=-4.7 E_L$
			
		}
		\label{fig:Figure3}
	\end{center}
\end{figure*}

To calculate the effective mass we fitted sinusoids to the sloshing motion of our atoms in the dipole trap and extracted the frequency of oscillation. Our Raman beams are co-propagating with the optical dipole trap beams, therefore the direction of the measured dipole trap frequencies is at a $45^{\circ}$ angle with respect to the direction of $k_R$. The kinetic and harmonic terms in the Hamiltonian are

\begin{equation}
\hat{H}_{\perp}=\frac{1}{2m^{\star}}k_x^2 + \frac{1}{2m}k_y^2+\frac{m}{2}[\omega_{x'}^2x'^2m+\omega_y'^2y'^2]
\end{equation}
%
with $\mathbf{e}_{x'}=\frac{\mathbf{e}_{x}+\mathbf{e}_{y}}{\sqrt{2}}$ and  $\mathbf{e}_{y'}=\frac{\mathbf{e}_{x}-\mathbf{e}_{y}}{\sqrt{2}}$. For $\omega_{x'}=\omega_{y'}$ a simple rotation yields a trapping frequency along the Raman direction $\omega_x=\sqrt{\omega_x^2+\omega_y^2}$.   
Figure \ref{fig:Figure4} shows the dipole oscillations of our BECs along the $\mathbf{e}_{x'}$ and $\mathbf{e}_{x'}$ directions for the three different coupling regimes we are studying, as well as the bare state motion. 

\begin{figure*}
	\begin{center}
		\includegraphics{figures/meff.pdf}
		\caption
		{
			{\bf a)} Top: Computed cross-sectional cuts of the total potential $V(y)$ (black), effective magnetic field ${\mathbfcal B}$ (blue) and atomic density $n(y)$ (yellow). Columns correspond to Raman coupling strengths $\hbar = 0.5 , 3.5, 5.5 \:E_L$ respectively, all with the detuning gradient . Bottom: GPE-computed 2D density distributions $n(x,y)$ using the same parameters.			
			{\bf b)} Second-moment of the momentum distribution as a function of and  in our system. Regions with large moments represent large potential well separations that have a synthetic magnetic field localized between the BECs.
			
		}
		\label{fig:Figure4}
	\end{center}
\end{figure*}


We use a not-uniform fast Fourier transform algorithm (NUFFT) which allows us to get the power spectral density for data points that are not necessarily evenly spaced in time as required by regular FFT algorithms. In order to account for missing data points. Figure $nn+2$ shows the power spectral density (PSD) of the time evolution of each $m_f$ state. Each vertical cut is normalized to the highest peak of the three spin states. We then rescale the units $\hbar\Delta\rightarrow	 \hbar^2k_x/8m$, extract the highest peaks in the PSD and offset their frequency by $\hbar^2k_x^2/2m^{\star}$, finally obtaining the characteristic spin- and momentum- dependent dispersion of a spin orbit coupled system. For the case driven spin-orbit coupling (cases (ii) and (iii)), since the strength of the drive is comparable to the quadratic Zeeman shift,we observe higher frequency components in the dynamics which are in good agreement with Floquet theory.

Our system has dark states at $\Delta=0$ which can be noted by the missing peaks in the PSD. Since there are eigenstates of the Raman dressed Hamiltonian that never get populated, the time evolution of the system does not have the frequency components related to the missing eigenstates. The presence of the dark states in the system is in good agreement with theory. 


\section{Discussion}

Heating problems? Talk about  Mention the possibility of measuring the Floquet bands. Possibility to extend this to the regime where cyclic coupling are not negligible and do butterfly physics.


Higher coupling strength compared to quadratic zeeman shift means more coupling within different Floquet manifolds. So interpreting the 'tuned' bands becomes more challenging.
Think about the coupling strength vs quadratic zeeman shift in terms of excited floquet states. 
Spectroscopy of bloch bands?
Does driving strength exceed transition frequency? No, but it does exceed the quadratic zeeman shift. 
From the floquet paper: The observed  system dynamics is very well described in terms of
quasienergies and quasienergy states, as predicted by
Floquet theory. In particular, we observe several frequency
components in the dynamics, in very good agreement with
theory

Heating due to scattering of spontaneously emitted photons is always present in our system. The time scales of our pulsing experiments never exceeded 1 ms which is small compared to the lifetime of our system (measure lifetime with and without Raman?). Heating is also present in periodically driven systems, and while it can be minimized by increasing the driving frequency one in exchange requires more power to achieve the same tunability in the system. 

%\section{Conclusion}

In conclusion, we can measure the spin and momentum dependent dispersion relation for a spin-1 spin-orbit coupled BEC using our Fourier spectroscopy thechnique. This method is good for any (effective) three level system with a quadratic Zeeman shift $\epsilon>4E_L$ and does not require any additional hardware as it relies only on the Hamiltonian evolution of the system. Our technique can also be useful to measure the Floquet quasi energy spectrum and the coupling within different Floquet manifolds for driven systems. 

%\section{Appendix A}


\subsection{Effective SOC Hamiltonian}	
%Probably too much. 
The effective spin-orbit coupling Hamiltonian can be derived from the electric dipole Hamiltonian describing the interaction between our Rb atoms and the multiple frequency Raman lasers


\begin{align}
	\hat{H}_{AL}(t) = \left[\Omega_{21}\cos(2\kr x-\omega_{21} t +\phi_1)+\Omega_{31}\cos(2\kr x-\omega_{31} t+\phi_2)+\Omega_{41}\cos(2\kr x-\omega_{41} t + \phi_3)\right]\hat{F_x}
\end{align}
%
where $\Omega_{ij}\propto \vec{E_i}\times\vec{E_j^{\star}}$ represents the coupling strength associated to each pair of Raman beams and $\omega_{ij} = \omega_{i}-\omega_{j} $. We choose the frequencies so that $\omega_{31} + \omega{21}$ is at 4 photon resonance with the $m_f = +1\rightarrow m_f = -1$ transition. We then apply a rotation about the z axis $\hat{U} = e^{i\bar{\omega} t\hat{F}_z}$ where $\bar{\omega} = \frac{\omega_{21}+\omega_{31}}{2}$. For the choice of parameters 
$\Omega_{21} =\Omega_{31} = \Omega $, $\Omega_{41}=\Omega_0$, and $\omega_{41} =\frac{\omega_{21}+\omega_{31}}{2}$,and after applying the rotating wave approximation (RWA), the Hamiltonian transforms to  
%
%

	
\begin{align}
	\begin{split}
		\hat{\tilde{H}}_{AL}(t) &= \frac{1}{2}\lbrace\Omega\cos(2\kr x+\delta\omega t +\phi_1)+\Omega\cos(2\kr x-\delta\omega t + \phi_2) +\Omega_0\cos(2\kr x + \phi_3)\rbrace \hat{F}_x \\
		& -  \frac{1}{2}\lbrace\Omega\sin(2\kr x+\delta\omega t +\phi_1)+\Omega\sin(2\kr x-\delta\omega t +\phi_2)+\Omega_0\sin(2\kr x +\phi_3)\rbrace \hat{F}_y
	\end{split}
\end{align}

we have the freedom of defining the time origin so we can get rid of one phase, lets make it $\phi_3$ for convenience. Can I get rid of a second phase? Make $\phi_1 = -\phi_2$
 
 %
 %
\begin{align}
 	\begin{split}
 		\hat{\tilde{H}}_{AL}(t)  = [\frac{\Omega_0}{2} + \Omega\cos(\delta\omega t+\phi_1)][\cos(2\kr x)\hat{F}_x - \sin(2\kr x)\hat{F}_y],
 	\end{split}
\end{align}


where we have defined $\delta\omega=\frac{\omega_{31}-\omega_{21}}{2}$. This Hamiltonian term describes a helically precessing effective Zeeman field with amplitude oscillating periodically in time. 

The complete Hamiltonian can therefore be written as

\begin{align}
	\begin{split}
		\hat{H}(t)=\frac{\hbar^2}{2m}\hat{k}^2 + \mu\mathbf{B}_{eff}\cdot\hat{\mathbf{F}}
		\label{Eq:Beff}
	\end{split}
\end{align}

(check units here)
with $\mu\mathbf{B}_{eff}=(\frac{\Omega_0}{2} + \Omega\cos(\delta\omega t+\phi_1))(\cos(2\kr x)\mathbf{e}_x-\sin(2\kr x)\mathbf{e}_y)+\Delta_0\mathbf{e}_z$


We can apply a position dependent rotation $\hat{U} = e^{i2\kr x \hat{F}_z}$ transforms our Hamiltonian into the form of Eq. \ref{Eq:SOCone} with a time dependent Raman coupling.

\begin{align}
	\begin{split}
		\hat{H}(t) = & \frac{\hbar^2}{2m}(\hat{k}-2\kr\hat{F}_z)^2 + (\frac{\Omega_0}{2} + \Omega\cos(\delta\omega t))\hat{F}_x \\& + \epsilon(\hat{F}_z^2-\mathbb{I})+\Delta\hat{F}_z \\
		=&\frac{\hbar^2\hat{k}^2}{2m} + \alpha_0\hat{k}\hat{F}_z +4E_L\mathbb{I} + \frac{\Omega(t)}{2}\hat{F}_x\\
		& +(\epsilon+4E_L)(\hat{F}_z^2-\mathbb{I}) +\Delta_0\hat{F}_z 
	\end{split}
\end{align}

(check factors of 2!)

To get rid of the time dependence in the Hamiltonian and ultimately getting the `tunable' spin-orbit coupling we can choose a transoformation of the Hamiltonian such that $\hat{U}^{\dagger} \frac{\partial\hat{U}}{\partial t} = -i \frac{\Omega(t)}{2}\hat{F}_x$. This will be satisfied for

\begin{align}
	\hat{U} = e^{-i\frac{\Omega}{2}\int_0^t\cos(\delta\omega t')dt'} = e^{-i\frac{\Omega}{2\delta\omega}\sin(\delta\omega t)}.
\end{align}

Under this time dependent transformation, the time evolution of the system will be given by the Hamiltonian
%

\begin{align}
	\begin{split}
		\hat{\tilde{H}} = & \hat{U}^{\dagger}\hat{H}(t)\hat{U} + i\hat{U}^{\dagger} \frac{\partial\hat{U}}{\partial t} \\
		 = &\frac{\hbar^2\hat{k}^2}{2m} + \alpha\hat{k}\hat{F}_z +4E_L\mathbb{I} + \frac{\Omega_0}{2}\hat{F}_x \\
		&+ \frac{\tilde{\Omega}}{2}\hat{F}_{xz} +(\tilde{\epsilon}+4E_L)(\hat{F}_z^2-\mathbb{I}) +\Delta\hat{F}_z 	
	\end{split}
\end{align}
%
%
which is exactly Eq. \ref{Eq:SOCeff}. To arrive to this final form we have transformed the operators that don't commute with $\hat{F}_x$ as

%
\begin{align}
	\begin{split}
		e^{i\theta \hat{F}_x} \hat{F}_z e^{-i\theta \hat{F}_x}=& \cos\theta \hat{F}_z + \sin\theta\hat{F}_y \\
		e^{i\theta \hat{F}_x} \hat{F}_y e^{-i\theta \hat{F}_x} =& -\sin\theta\hat{F}_z +\cos\theta\hat{F}_y \\
		e^{i\theta \hat{F}_x} \hat{F}_z^2 e^{-i\theta \hat{F}_x} = &\cos^2\theta\hat{F}_z^2+\sin^2\theta\hat{F}_y^2 + \sin\theta\cos\theta(\hat{F}_z\hat{F}_y + \hat{F}_y\hat{F}_z).
	\end{split}
\end{align}
%
and neglected the terms oscillating at high frequency

\begin{align}
	\begin{split}
	\cos(\Omega/2\delta\omega\sin(\delta\omega t))&= J_0(\Omega/2\delta\omega) + 2\sum_{n=1}^{\infty}J_{2n}(\Omega/2\delta\omega)\cos(2n(\delta\omega t)) \\
	& \approx J_0(\Omega/2\delta\omega) \\
	\sin(\Omega/2\delta\omega\sin(\delta\omega t))&= 2\sum_{n=0}^{\infty}J_{2n+1}(\Omega/2\delta\omega)\sin((2n+1)(\delta\omega t)) \approx 0,
	\end{split}
\end{align} 


(verify how large $\delta\omega$ needs to be for this approximation to be valid)
%
%
 







Fixes:
\begin{itemize}
	\item Write frequencies as $\omega_L$ and $\omega_L +\Delta\omega \pm\delta\omega$. Is there too many $\delta$ symbols, confusing?
	\item Call $J_0$ the zeroth order Bessel functions of the first kind.
	\item don't say modulation or multiple frequencies, just say we amplitude modulate by using 
	\item no effective model, say the floquet Hamiltonian takes de form
	\item mention 2 paths, aka Molmer?
	\item say something about spin one paper?
	\item sumar masa efectiva y comparar con 
	\item 
\end{itemize}




Corrections of the effective Hamiltonian are of the order $1/\delta\omega$.

%% Here is the endmatter stuff: Supplementary Info, etc.
%% Use \item's to separate, default label is "Acknowledgments"

%\begin{addendum}
%\item We appreciate useful discussions with Stefan Natu.  This work was partially supported by the ARO's atomtronics MURI, by the AFOSR's Quantum Matter MURI, NIST, and the NSF through the PFC at the JQI.
 
%\item[Author Contributions] All authors excepting I.B.S contributed to the data taking effort.  All authors: analyzed data; performed numerical and analytical calculations; and contributed to writing the manuscript.  I.B.S. proposed the initial experiment (with great enthusiasm, but with little appreciation for the many technical hurdles to be resolved by the remainder of the team).

%\item[Correspondence] Correspondence and requests for materials should be addressed to I.B.S.~ (email: ian.spielman@nist.gov).

%\bibliography{MolmerSorensen}

\end{document}
