\section{Discussion}


Spectroscopy of bloch bands?

Heating due to scattering of spontaneously emitted photons is always present in our system. It is also well known that heating is present in periodically driven systems, and while it can be minimized by increasing the driving frequency one in exchange requires more Raman power to tune the spin-orbit coupling strength. The time scales of our pulsing experiments never exceeded 1 ms which is small compared to the lifetime of our system.

In conclusion, we can measure the spin and momentum dependent dispersion relation for a spin-1 spin-orbit coupled BEC using our Fourier spectroscopy thechnique. This method is good for any (effective) three level system with a quadratic Zeeman shift $\epsilon>4E_L$ and does not require any additional hardware as it relies only on the Hamiltonian evolution of the system. Our technique can also be useful to measure the Floquet quasi energy spectrum and the coupling within different Floquet manifolds for driven systems. 