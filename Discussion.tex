\section{Discussion}


Heating due to scattering of spontaneously emitted photons is always present in our system. It is also well known that heating is present in periodically driven systems, and while it can be minimized by increasing the driving frequency, it in exchange requires more Raman power to tune the spin-orbit coupling strength. The time scales of our pulsing experiments never exceeded $900\us$ which is small compared to the lifetime of our system.

In conclusion, we can measure the spin and momentum dependent dispersion relation for a spin-1 spin-orbit coupled BEC using our Fourier spectroscopy technique. When used to study periodically driven systems, we are able tp see a rich spectrum that arises from the Floquet quasi-energies. This method allows one to measure the spectrum of a system just by looking at the time evolution without the need of fitting to a complicated model and is good for any effective three level system with a quadratic branch in the spectrum. This technique might prove particularly useful to probe the spin-resolved energy dispersion of atoms in the presence of Rashba spin-orbit coupling using newly proposed schemes to generate this type of coupling without the use of excited states  \cite{_rashba_????} and could lead to a better understanding of new topological materials.
