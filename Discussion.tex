\section{Discussion}

Heating problems? Talk about  Mention the possibility of measuring the Floquet bands. Possibility to extend this to the regime where cyclic coupling are not negligible and do butterfly physics.


Higher coupling strength compared to quadratic zeeman shift means more coupling within different Floquet manifolds. So interpreting the 'tuned' bands becomes more challenging.
Think about the coupling strength vs quadratic zeeman shift in terms of excited floquet states. 
Spectroscopy of bloch bands?
Does driving strength exceed transition frequency? No, but it does exceed the quadratic zeeman shift. 
From the floquet paper: The observed  system dynamics is very well described in terms of
quasienergies and quasienergy states, as predicted by
Floquet theory. In particular, we observe several frequency
components in the dynamics, in very good agreement with
theory

Heating due to scattering of spontaneously emitted photons is always present in our system. The time scales of our pulsing experiments never exceeded 1 ms which is small compared to the lifetime of our system (measure lifetime with and without Raman?). Heating is also present in periodically driven systems, and while it can be minimized by increasing the driving frequency one in exchange requires more power to achieve the same tunability in the system. 