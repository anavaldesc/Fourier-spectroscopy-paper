\section{Results}

We mapped the three band structure of our spin-orbit coupled atoms. We fit the data to the Hamiltonian in Eq. \ref{Eq:SOCone} with $\Delta_0$ and $\Omega_R$ as the only free parameters. The fitted values agree well with microwave and Raman power calibrations. Figure \ref{fig:Figure3} shows representative traces for the time evolution of our system for the three cases outlined above. The Raman coupling strength remained nominally constant throughout our measurements. 

\begin{figure*}
	\begin{center}
		\includegraphics{figures/tdse.pdf}
		\caption 
		{
		Time evolution of the BEC for Raman pulsing times between 5 and 10 $\mu s$, for different spin orbit coupling regimes:
		{\bf (i)} $\Omega_0=9.9 E_L$, $\Omega=04$,  $\Delta=5.8 E_L$, 
		{\bf (ii)} $\Omega_0=0$, $\Omega=8.6 E_L$,  $\Delta=-0.7 E_L$, and
		{\bf (iii)} $\Omega_0=1.5 E_L$, $\Omega=8.4 E_L$,  $\Delta=-4.7 E_L$
			
		}
		\label{fig:Figure3}
	\end{center}
\end{figure*}

Since the Raman coupling strength $\Omega, \Omega_R$ is comparable to the separation between the Floquet manifolds $\delta\omega$, the time evolution shows higher frequency components for the driven SOC measurements (cases (ii) and (iii)), that correspond to the Floquet quasi-energy spectrum,  

We use a non-uniform fast Fourier transform algorithm (NUFFT) to obtain the power spectral density since our data points are not evenly spaced in time due to experimental uncertainties. Figure \ref{fig:Figure4} shows the power spectral density (PSD) of the time evolution of each $m_F$ state. Each vertical cut is normalized to the highest peak of the three spin states.
	
\begin{figure*}
	\begin{center}
		\includegraphics{figures/bands.pdf}
		\caption 
		{
			Power spectral density of the experimentally measured probability amplitudes for for different spin orbit coupling regimes:
			{\bf (i)} $\Omega_0=9.9 E_L$, $\Omega=04$,  $\Delta=5.8 E_L$, 
			{\bf (ii)} $\Omega_0=0$, $\Omega=8.6 E_L$,  $\Delta=-0.7 E_L$, and
			{\bf (iii)} $\Omega_0=1.5 E_L$, $\Omega=8.4 E_L$,  $\Delta=-4.7 E_L$
			
		}
		\label{fig:Figure4}
	\end{center}
\end{figure*}

For the second and third rows we notice peaks at constant frequencies of $\delta\omega$ and $2\delta\omega$, independently of the Raman detuning, and a structure that is symmetric about the frequencies $2\pi f=\delta\omega/2$ and $2\pi f=\delta\omega$  which can be interpreted as the atoms coupling to the neighboring Floquet manifolds.  

The missing peaks in the PSD are due to dark states at $\Delta_0=0$. Since there are eigenstates of the Raman dressed Hamiltonian that never get populated, the time evolution of the system does not have the frequency components related to the missing eigenstates. 

The effective mass was extracted by comparing the sloshing frequencies of the atoms in the dipole trap to the Raman dressed ones.  fitted sinusoids to the sloshing motion of our atoms in the dipole trap and extracted the frequency of oscillation. % Our Raman beams are co-propagating with the optical dipole trap beams, therefore the direction of the measured dipole trap frequencies is at a $45^{\circ}$ angle with respect to the direction of $k_L$. The kinetic and harmonic terms in the Hamiltonian are

\begin{equation}
\hat{H}_{\perp}=\frac{1}{2m^{\star}}k_x^2 + \frac{1}{2m}k_y^2+\frac{m}{2}[\omega_{x'}^2x'^2m+\omega_y'^2y'^2]
\end{equation}
%
with $\mathbf{e}_{x'}=\frac{\mathbf{e}_{x}+\mathbf{e}_{y}}{\sqrt{2}}$ and  $\mathbf{e}_{y'}=\frac{\mathbf{e}_{x}-\mathbf{e}_{y}}{\sqrt{2}}$. For $\omega_{x'}=\omega_{y'}$ a simple rotation yields a trapping frequency along the Raman direction $\omega_x=\sqrt{\omega_x^2+\omega_y^2}$.   
Figure \ref{fig:Figure5} shows the dipole oscillations along the $\mathbf{e}_{x'}$ and $\mathbf{e}_{x'}$ directions for the three different coupling regimes we are studying, as well as the bare state motion. 

\begin{figure*}
	\begin{center}
		\includegraphics{figures/meff.pdf}
		\caption
		{  Oscillation of the BEC in the dipole trap along the directions $\mathbf{e}_{x'}$ and  $\mathbf{e}_{y'}$ defiend by the propagation of the dipole tap beams. The traces have been shifted so that it is easier to appreciate the change in the motion for each coupling regime.	
		}
		\label{fig:Figure5}
	\end{center}
\end{figure*}

We can obtain the characteristic dispersion of a SOC system after subtracting the effective mass, as can be seen in Figure \ref{fig:Figure6}. The measured bands are in good agreement with the eigenvalues calculated from Floquet theory. 


\begin{figure*}
	\begin{center}
		\includegraphics{figures/bands2.pdf}
		\caption 
		{
			This figure needs some love but this is more or less the idea. Only the first panel has the effective mass substracted, other 2 are energy differences. I was thinking of maybe making the theory lines thicker to include uncertainties. 
			
		}
		\label{fig:Figure6}
	\end{center}
\end{figure*}
