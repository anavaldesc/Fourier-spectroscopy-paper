\section{Results}

We studied the time evolution for three different schemes of SOC. 
We can fit our experimentally calculated probability amplitudes to the Hamiltonian evolution. This can serve two purposes

We calibrated the Raman coupling terms $\Omega$ and $\Omega_0$ and the detuning from Raman resonance $\Delta$ by fitting the three-level Rabi oscillations of the $\ket{m_F=0}$ and $\ket{m_f=\pm 1}$ states to the time evolution given by the Hamiltonian in Eq. \ref{Eq:SOCone}. Figure \ref{fig:Figure3} shows representative traces for the time evolution of our system for the three cases outlined above. These calibrations along with the information gained from the images of atoms out-coupled  with microwaves guaranteed that we were indeed at the correct detuning from Raman resonance and that the Raman coupling strength remained nominally constant throughout our measurements. 

\begin{figure*}
	\begin{center}
		\includegraphics{figures/tdse.pdf}
		\caption 
		{
		Time evolution of the BEC for Raman pulsing times between 5 and 10 $\mu s$, for different spin orbit coupling regimes:
		{\bf (i)} $\Omega_0=9.9 E_L$, $\Omega=04$,  $\Delta=5.8 E_L$, 
		{\bf (ii)} $\Omega_0=0$, $\Omega=8.6 E_L$,  $\Delta=-0.7 E_L$, and
		{\bf (iii)} $\Omega_0=1.5 E_L$, $\Omega=8.4 E_L$,  $\Delta=-4.7 E_L$
			
		}
		\label{fig:Figure3}
	\end{center}
\end{figure*}



The time evolution shows higher frequency components for cases (ii) and (iii), as expected from the Floquet quasi-energy spectrum, since the Raman coupling strength $\Omega, \Omega_R$ is comparable to the separation between the Floquet manifolds $\epsilon$. 


\begin{figure*}
	\begin{center}
		\includegraphics{figures/bands.pdf}
		\caption 
		{
			Time evolution of the BEC for Raman pulsing times between 5 and 10 $\mu s$, for different spin orbit coupling regimes:
			{\bf (i)} $\Omega_0=9.9 E_L$, $\Omega=04$,  $\Delta=5.8 E_L$, 
			{\bf (ii)} $\Omega_0=0$, $\Omega=8.6 E_L$,  $\Delta=-0.7 E_L$, and
			{\bf (iii)} $\Omega_0=1.5 E_L$, $\Omega=8.4 E_L$,  $\Delta=-4.7 E_L$
			
		}
		\label{fig:Figure3}
	\end{center}
\end{figure*}

To calculate the effective mass we fitted sinusoids to the sloshing motion of our atoms in the dipole trap and extracted the frequency of oscillation. Our Raman beams are co-propagating with the optical dipole trap beams, therefore the direction of the measured dipole trap frequencies is at a $45^{\circ}$ angle with respect to the direction of $k_R$. The kinetic and harmonic terms in the Hamiltonian are

\begin{equation}
\hat{H}_{\perp}=\frac{1}{2m^{\star}}k_x^2 + \frac{1}{2m}k_y^2+\frac{m}{2}[\omega_{x'}^2x'^2m+\omega_y'^2y'^2]
\end{equation}
%
with $\mathbf{e}_{x'}=\frac{\mathbf{e}_{x}+\mathbf{e}_{y}}{\sqrt{2}}$ and  $\mathbf{e}_{y'}=\frac{\mathbf{e}_{x}-\mathbf{e}_{y}}{\sqrt{2}}$. For $\omega_{x'}=\omega_{y'}$ a simple rotation yields a trapping frequency along the Raman direction $\omega_x=\sqrt{\omega_x^2+\omega_y^2}$.   
Figure \ref{fig:Figure4} shows the dipole oscillations of our BECs along the $\mathbf{e}_{x'}$ and $\mathbf{e}_{x'}$ directions for the three different coupling regimes we are studying, as well as the bare state motion. 

\begin{figure*}
	\begin{center}
		\includegraphics{figures/meff.pdf}
		\caption
		{
			{\bf a)} Top: Computed cross-sectional cuts of the total potential $V(y)$ (black), effective magnetic field ${\mathbfcal B}$ (blue) and atomic density $n(y)$ (yellow). Columns correspond to Raman coupling strengths $\hbar = 0.5 , 3.5, 5.5 \:E_L$ respectively, all with the detuning gradient . Bottom: GPE-computed 2D density distributions $n(x,y)$ using the same parameters.			
			{\bf b)} Second-moment of the momentum distribution as a function of and  in our system. Regions with large moments represent large potential well separations that have a synthetic magnetic field localized between the BECs.
			
		}
		\label{fig:Figure4}
	\end{center}
\end{figure*}


We use a not-uniform fast Fourier transform algorithm (NUFFT) which allows us to get the power spectral density for data points that are not necessarily evenly spaced in time as required by regular FFT algorithms. In order to account for missing data points. Figure $nn+2$ shows the power spectral density (PSD) of the time evolution of each $m_f$ state. Each vertical cut is normalized to the highest peak of the three spin states. We then rescale the units $\hbar\Delta\rightarrow	 \hbar^2k_x/8m$, extract the highest peaks in the PSD and offset their frequency by $\hbar^2k_x^2/2m^{\star}$, finally obtaining the characteristic spin- and momentum- dependent dispersion of a spin orbit coupled system. For the case driven spin-orbit coupling (cases (ii) and (iii)), since the strength of the drive is comparable to the quadratic Zeeman shift,we observe higher frequency components in the dynamics which are in good agreement with Floquet theory.

Our system has dark states at $\Delta=0$ which can be noted by the missing peaks in the PSD. Since there are eigenstates of the Raman dressed Hamiltonian that never get populated, the time evolution of the system does not have the frequency components related to the missing eigenstates. The presence of the dark states in the system is in good agreement with theory. 
